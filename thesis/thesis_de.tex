\documentclass[twoside,a4paper,fleqn,12pt]{article}
\usepackage{fancyhdr,a4wide,graphicx}
\usepackage[paper=a4paper,left=20mm,right=20mm,top=25mm,bottom=25mm]{geometry}
\usepackage{ngerman}
\pagestyle{fancy}
\usepackage[utf8]{inputenc}
\usepackage[]{amsfonts}
\usepackage{amsmath}
\usepackage{color}
\usepackage{colortbl}
\usepackage{mathptmx}
\usepackage[scaled=.90]{helvet}
\usepackage{courier}
\usepackage{verbatim}
\usepackage{sectsty}
\usepackage{listings}
\usepackage{scalefnt}
\usepackage{setspace}

\definecolor{darkred}{rgb}{.5,0,0}
\definecolor{darkblue}{rgb}{0,0,.5}
\usepackage[plainpages=false,pdfpagelabels,colorlinks=true,urlcolor=darkblue,pagecolor=darkred,citecolor=darkred,linkcolor=darkred]{hyperref}
%\newcommand\url[1]{\texttt{#1}}

\lstset{basicstyle=\ttfamily\small,lineskip=-0.5em}

% define the title
\author{Frank Richter 68278\\frank.richter@gmail.com}
\title{\usefont{OT1}{phv}{b}{n}\selectfont Entwicklung eines Compilers für eine auf Cg basierende Sprache zur Programmierung von Graphikkarten \normalfont}
\date{\today}

\begin{document}

\newcommand\btxandlong{und}
\newcommand\btxandshort{u}
\newcommand\Btxinlong{In}
\newcommand\Btxinshort{I}
\newcommand\btxpageslong{Seiten}
\newcommand\btxetalshort{et al}
\newcommand\btxeditionlong{Auflage}
\bibliographystyle{mystyle}

% Zeilenabstand 1.5
\renewcommand{\baselinestretch}{1.50}\normalsize

% Helvetoca für Section-Titel
\allsectionsfont{\usefont{OT1}{phv}{b}{n}\selectfont}

% Different font in captions
\newcommand{\captionstyle}{\small\centering}

\makeatletter  % Allow the use of @ in command names
\long\def\@makecaption#1#2{%
  \vskip\abovecaptionskip
  \sbox\@tempboxa{{\captionstyle #1: #2}}%
  \ifdim \wd\@tempboxa >\hsize
    {\captionstyle #1: #2\par}
  \else
    \hbox to\hsize{\hfil\box\@tempboxa\hfil}%
  \fi
  \vskip\belowcaptionskip}
\makeatother   % Cancel the effect of \makeatletter

% Fussnoten: alle Zeilen einrücken
\makeatletter
\newlength{\myFootnoteWidth}
\newlength{\myFootnoteLabel}
\setlength{\myFootnoteLabel}{1.2em}%  <-- can be changed to any valid value
\renewcommand{\@makefntext}[1]{%
  \setlength{\myFootnoteWidth}{\columnwidth}%
  \addtolength{\myFootnoteWidth}{-\myFootnoteLabel}%
  \noindent\makebox[\myFootnoteLabel][r]{\@makefnmark\ }%
  \parbox[t]{\myFootnoteWidth}{#1}%
}
\makeatother

% ---------- normal title ---------- %
\titlepage
\maketitle
\thispagestyle{empty}
\newpage
\thispagestyle{empty}
\mbox{}

% ---------- Fancyheader ---------- %
\fancyhead[L]{}
\fancyhead[R]{}
\fancyfoot[C]{\today}
\fancyfoot[R]{\footnotesize \thepage{}}
\renewcommand{\headrulewidth}{0pt}
\setlength{\headheight}{24pt}

% generates the title


% ---------- table of contents ---------- %
\newpage
\pagenumbering{roman}
%\addcontentsline{toc}{section}{Inhaltsverzeichnis}
\pdfbookmark[1]{Inhaltsverzeichnis}{myPDFtocLabel}
\tableofcontents

\cleardoublepage
\pagenumbering{arabic}
\newcommand\todo[1]{\footnote{\textcolor{red}{TODO: #1}}}

\section{Zielstellung}

\section{Sprachspezifikation}

Dieser Abschnitt beschreibt konkret eine Ein-Programm-Shadingsprache und geht zuerst auf Eckpunkte der Syntax und danach
auf allgemeinere Aspekte (z.B. wie die Mischung von Operationen verschiedener Berechnungsfrequenzen in verschiedenen Ausführungszweigen) ein.

\subsection{Anforderungen an die Sprache}

Auf höherer Ebene soll die Sprache folgenden Anforderungen erfüllen:
\begin{itemize}
\item \emph{C-ähnliche Syntax}: Um ein schnelles Einarbeiten und Verständnis zu ermöglichen soll sich die Syntax an der der verbreiteten
Shadingsprachen Cg, DirectX HLSL und OpenGL GLSL orientieren. Diese wiederum sind C-verwandt.
Weitere syntaktische Anleihen können auch von passenden Stellen der anderen besprochenen  Shadingsprachen übernommen werden.
\item \emph{Moderne Entwicklungen berücksichtigen}: Die neuesten der C-ähnlichen Sprachen sind Java und C\#. Wenn angebracht sollen
Konzepte aus diesen übernommen werden.
\item \emph{Vereinfachungen für Shading}: Die Sprache soll Elemente besitzen, die das Erstellen von Shading-Programmen
vereinfachen. \item \emph{Keine Bindung an bestimmte Architektur}: Die Sprache selbst soll keine bestimmte darunterliegende Hardwarearchitektur
vorraussetzen; auch keine Architektur-Klasse wie z.B. "`Grafikprozessoren"'\footnote{Sinnvollerweise wird eine von-Neumann-Architektur minimal vorausgesetzt.}.
Auch soll die Sprache insgesamt weniger hardwarenah als C sein.
\end{itemize}

% Mehr?

Aus diesen Anforderungen leiten sich die folgenden spezifischen Kriterien ab:

\begin{itemize}
\item \emph{Syntax}: Die Syntax soll im Wesentlichen auf C basieren. Es sollen Elemente ausgelassen werden, die eher 
unintuitiv sind und selten praktisch verwendet werden, wie die Oktalnotation bei Integerliteralen.
Syntaxerweiterungen sollten Inkonsistenzen oder Auslassungen betreffen - so gibt es bei C z.B. 
nur bei der Definition einer Array-Variable die Möglichkeit, dieser mehrere Element auf einmal zuzuweisen.
Bei einer regulären Zuweisung ist dies bei C nicht möglich, soll aber in der neuen Sprache ermöglicht werden.
\item \emph{Zeichensatz}: ANSI C basiert auf dem ASCII-Zeichensatz. % TODO Ref
Mit Unicode gibt es es einen Zeichensatz, der alle weltweit geschriebenen Sprachen umfasst; dieser
hat auch in den Definitionen von Programmiersprachen Einzug gehalten % TODO Ref C#
und erlaubt z.B. Bezeichner in der Muttersprache des Programmierers zu deklarieren.
Aus technischer Sicht gibt es keinen Grund, nicht Unicode als Basis für den Zeichensatz zu benutzen.
\item \emph{Überdeckung von Bezeichnern}: Bei C\# können Bezeichner nicht anderer Bezeichner überdecken. Dies
verhindert eine Klasse von Fehlern, bei der durch Überdeckung nicht die Variable benutzt wird, die der
Programmierer angenommen hat. Das Verbot der Überdeckung soll übernommen werden.
\item \emph{Shadingspezifische Elemente}: Es sollen Vektor- und Matrixtypen vorhanden sein.
\item \emph{Keine Bindung an bestimmte Architektur}: Cg besitzt syntaktische Elemente, um bestimmte Variablen an
von der Grafikhardware bereitgestellte Werte zu binden. Auch gibt es einen Mechanismus zur Überladung von Funktionen
für verschiedene Zielhardware. Solche Elemente sollen \emph{nicht} in der neuen Sprache enthalten sein.
\item \emph{Maschinenferne}: Dynamische Verweise und 
Zeiger werden nicht unterstützt. Statische Verweise (z.B. für Ausgabevariablen bei Funktionen) sollten aber unterstützt
werden.
\end{itemize}

\newpage

\subsection{Allgemeine Aspekte und Probleme}

\subsubsection{Programmumgebung}

Ein Shaderprogramm läuft praktisch immer im Kontext einer Anwendung, die Parameter direkt oder indirekt (wie
Vertexdaten) an das Shaderprogramm übergibt.
Bei den Parametern der Hauptfunktion des Shaderprogramms, welche von diesem selbst nicht definiert werden,
wird deshalb davon ausgegangen, das diese von der umgebenden Anwendung spezifiziert werden.

\subsubsection{Definition Berechnungsfrequenz}
\label{Berechnungsfrequenz}

Die \emph{Berechnungsfrequenz eines Ausdrucks} beschreibt, wie oft sich der Wert des gegebenen Ausdrucks, auf alle Ausführungen eines
Programmes gesehen, ändert.

Die kleinstmögliche Frequenz besitzen statische Konstanten.

Die höchstmögliche Frequenz besitzen Ausdrücke, die bei jeder Auswertung einen anderen Wert liefern.
\emph{(Beispiel: Werte aus einer externen Datenquelle.)}

\subsubsection{Berechnungsfrequenzen bei Shading}

Bei Shadingsprachen sind folgende speziellen Berechnungsfrequenzen vorhanden:
\begin{itemize}
\item \emph{Mesh-Frequenz}: Ausdruck ist konstant während der Darstellung eines Gitternetzes, aber nicht zwischen
verschiedenen Gitternetzen oder Darstellungsläufen. \emph{(Beispiel: Position in Weltkoordinaten.)}
\item \emph{Vertex-Frequenz}: Ausdruck ändert sich von Vertex zu Vertex. \emph{(Beispiel: Texturkoordinaten.)}
\item \emph{Fragment-Frequenz}: Ausdruck ändert sich von Fragment zu Fragment. \emph{(Beispiel: Aus Textur gelesener Wert.)}
\end{itemize}

Als Besonderheit kommt hinzu, dass aus Daten mit \emph{Vertex-Frequenz} durch lineare Interpolation Daten mit
\emph{Fragment-Frequenz} gewonnen werden. \emph({Beispiel: für Auslesen einer Textur verwendete Koordinaten)}
Dies führt dazu, dass Ausdrücke, die nach den folgenden Regeln
eigentlich Fragment-Frequenz besitzen müssten, Vertex-Frequenz besitzen, unter der Bedingung, dass
das Ergebnis einer linearen Interpolation des Gesamtausdrucks äquivalent zum Ergebnis bei linearer
Interpolation aller Teilausdrücke (mit jeweils immer gleicher Gewichtung) ist.

\subsubsection{Formulierte Frequenz}

Zwar soll die Berechnungsfrequenz von Operationen soweit möglich automatisch bestimmt werden, trotzdem muss eine Frequenz
gewählt werden, in der alle Operationen vorerst formuliert werden.

Betrachtet man "`niedrige"' Frequenzen wie Berechnungen per Vertex, so erkennt man, dass sich damit Berechnungen höherer
Frequenz (also Fragmente) schlecht darstellen lassen. Betrachtet man Berechnungen auf Fragment-Ebene, so lassen sich
unter Umständen Berechnungen niedriger Frequenz ableiten (z.B. wenn beide Operatoren einer Operation nur von Vertexattributen
oder Konstanten abhängen). Shaderprogramme sollten also grundsätzlich in "`Fragmentfrequenz"' formuliert werden.

\subsubsection{Bestimmung der Berechnungsfrequenz}

\paragraph{Variablen:} Variablen besitzen eine zugeordnete Berechnungsfrequenz, die angibt wie oft sich der enthaltene Wert ändert.
Diese Frequenz entspricht der Frequenz des zuletzt zugewiesenen Ausdrucks (oder undefiniert wenn noch keine Zuweisung stattfand).

\paragraph{Arrays:} Ist die Größe eines Arrays statisch bekannt kann jedes Element als einzelne Variable gesehen werden.
Ist die Größe eines Arrays nicht statisch bekannt so muss für ein Element die höchste möglicherweise zugewiesene
Frequenz angenommen werden.

\paragraph{Ausdrücke:} Die Berechnungsfrequenz eines Wertes eines Ausdrucks ergibt sich aus der kleinsten gemeinsamen Frequenz
der in dem Ausdruck enthaltenen Teilausdrücke \emph{nach} möglicher Anwendung von Umformungen zur Vereinfachung/Optimierung --
sofern keine \emph{Spezialregeln} greifen (siehe Unten).

\subsubsection{Spezialregeln für Berechnungsfrequenzen von Ausdrücken}

\paragraph{Manuelle Bestimmung:}
Bei einigen Ausdrücken kann eine Frequenz nicht automatisch abgeleitet werden, insbesondere bei Eingabeparametern aus der
Umgebung. Es wird davon ausgegangen, dass die umgebende Anwendung dem Compiler mitteilt, mit welcher Frequenz ein Eingabeparameter
von der Anwendung geändert wird.\\

Die Beschaffenheit der Daten beim Shading erlaubt
weitere
%einige
Spezialfälle:
\paragraph{Lineare Operationen:} Vertexattribute werden über die Fragmente eines Dreiecks linear interpoliert; deswegen ist es
erlaubt, lineare Operationen auf Vertex-Frequenz zu vollziehen, wenn alle Operatoren mit Vertex-Frequenz oder darunter
berechnet wurden.

\paragraph{Spezielle Eigenschaften:} "`Spezialwissen"' über mögliche Werte kann benutzt werden, um die Frequenz eines
Ausdrucks zu bestimmen, insbesondere wenn diese niedriger als die normalerweise bestimme Frequenz wäre.
\begin{itemize}
\item \emph{Einheitsvektoren:} Einheitsvektoren werden oft in Shadingprogrammen benutzt. Einheitsvektoren können aber 
nicht linear interpoliert werden, da dies einen Verlust der Einheitslänge zur Folge haben kann. Trotzdem kann es effizienter
sein, einen Einheitsvektor mit Vertex-Frequenz zu berechnen, über die Fragmente linear zu interpolieren und per Fragment
zu normalisieren, als den Vektor vollständig mit Fragment-Frequenz zu berechnen.
\end{itemize}

\newpage
\section{Spezifikation}

\emph{Anmerkung:} 
Die Syntaxbeschreibung erfolgt auf der Ebene lexikalischer Einheiten. 
Deren Regeln sind in \ref{Lexikalische Einheiten} beschrieben. 

\newcommand\Gnt[1]{\emph{#1}}
\newcommand\Gt[1]{\texttt{#1}}
\newcommand\Gspecial[1]{#1}
\newcommand\attr[1]{\texttt{#1}}
\newcommand\kw[1]{\texttt{#1}}
\newcommand\op[1]{\glq\texttt{#1}\grq}
\newcommand\pmbr[1]{\texttt{#1}}
\newcommand\ident[1]{\texttt{#1}}
\newcommand\plus{+}
\newcommand\kwMlcStart{\kw{/*}}
\newcommand\kwMlcEnd{\kw{*/}}
\subsection{Programm}\label{Programm}
\Gnt{programm}:\label{programm}\\
\hspace*{1cm}\Gnt{programm\_statements}{\scriptsize (\ref{programm_statements})}  \Gnt{EOF}{\scriptsize (\ref{EOF})} \\
\Gnt{programm\_statements}:\label{programm_statements}\\
\hspace*{1cm}\Gnt{dekl\_funktion}{\scriptsize (\ref{dekl_funktion})} \\
\hspace*{1cm}\Gnt{dekl\_funktion}{\scriptsize (\ref{dekl_funktion})} \Gnt{programm\_statements}{\scriptsize (\ref{programm_statements})} \\
\hspace*{1cm}\Gnt{typ\_definition}{\scriptsize (\ref{typ_definition})}  \glq\Gt{;}\grq \\
\hspace*{1cm}\Gnt{typ\_definition}{\scriptsize (\ref{typ_definition})}  \glq\Gt{;}\grq \Gnt{programm\_statements}{\scriptsize (\ref{programm_statements})} \\
\hspace*{1cm}\Gnt{dekl\_var}{\scriptsize (\ref{dekl_var})}  \glq\Gt{;}\grq \\
\hspace*{1cm}\Gnt{dekl\_var}{\scriptsize (\ref{dekl_var})}  \glq\Gt{;}\grq \Gnt{programm\_statements}{\scriptsize (\ref{programm_statements})} \\
\hspace*{1cm}\Gnt{dekl\_konst}{\scriptsize (\ref{dekl_konst})}  \glq\Gt{;}\grq \\
\hspace*{1cm}\Gnt{dekl\_konst}{\scriptsize (\ref{dekl_konst})}  \glq\Gt{;}\grq \Gnt{programm\_statements}{\scriptsize (\ref{programm_statements})} \\


Ein \emph{Programm} besteht aus jeweils keinen bis mehreren Typdefinitionen, Variablen-, Konstanten- und Funktionsdeklarationen.
Der Gültigkeitsbereich dieser Deklarationen ist global.

\label{EOF} Das Terminal \emph{EOF} markiert das Ende der Eingabe.


\subsubsection{{Eintrittsfunktion}}\label{__Eintrittsfunktion__}
Der Eintrittspunkt der Ausführung eines Programms ist eine Funktion~(\ref{Funktionen}) namens "`\ident{main}"'. Ist eine solche Funktion nicht definiert, so
kann das Programm nicht ausgeführt werden.

Die Eintrittsfunktion muss den Rückgabetyp \kw{void} besitzen.

Es müssen genau zwei Ausgabeparameter vom Typ \kw{float4}~(siehe \Gnt{typ\_vektor\_float}{\scriptsize (\ref{typ_vektor_float})}) deklariert werden.
Der erste Ausgabeparameter nimmt die transformierte und projizierte Position entgegen.
Der zweite Ausgabeparameter nimmt die auszugebende Fragmentfarbe entgegen.

Es können beliebig viele Eingabeparameter beliebigen Typs für die Eintrittsfunktion deklariert werden.
Die Werte dieser Parameter entstammen der Programmumgebung.


\subsection{Blöcke}\label{Bloecke}
\Gnt{block}:\label{block}\\
\hspace*{1cm}\Gnt{typ\_definition}{\scriptsize (\ref{typ_definition})} \\
\hspace*{1cm}\Gnt{typ\_definition}{\scriptsize (\ref{typ_definition})} \Gnt{block}{\scriptsize (\ref{block})} \\
\hspace*{1cm}\Gnt{dekl\_var}{\scriptsize (\ref{dekl_var})}  \glq\Gt{;}\grq \\
\hspace*{1cm}\Gnt{dekl\_var}{\scriptsize (\ref{dekl_var})}  \glq\Gt{;}\grq \Gnt{block}{\scriptsize (\ref{block})} \\
\hspace*{1cm}\Gnt{dekl\_konst}{\scriptsize (\ref{dekl_konst})}  \glq\Gt{;}\grq \\
\hspace*{1cm}\Gnt{dekl\_konst}{\scriptsize (\ref{dekl_konst})}  \glq\Gt{;}\grq \Gnt{block}{\scriptsize (\ref{block})} \\
\hspace*{1cm}\Gnt{kommando}{\scriptsize (\ref{kommando})} \\
\hspace*{1cm}\Gnt{kommando}{\scriptsize (\ref{kommando})} \Gnt{block}{\scriptsize (\ref{block})} \\
\Gnt{kommando}:\label{kommando}\\
\hspace*{1cm}\Gnt{ausdruck}{\scriptsize (\ref{ausdruck})}  \glq\Gt{;}\grq \\
\hspace*{1cm}\glq\Gt{return}\grq  \glq\Gt{;}\grq \\
\hspace*{1cm}\glq\Gt{return}\grq \Gnt{ausdruck}{\scriptsize (\ref{ausdruck})}  \glq\Gt{;}\grq \\
\hspace*{1cm}\Gnt{verzweigung}{\scriptsize (\ref{verzweigung})} \\
\hspace*{1cm}\Gnt{schleife\_for}{\scriptsize (\ref{schleife_for})} \\
\hspace*{1cm}\Gnt{schleife\_while}{\scriptsize (\ref{schleife_while})} \\
\hspace*{1cm}\glq\Gt{\{}\grq  \glq\Gt{\}}\grq \\
\hspace*{1cm}\glq\Gt{\{}\grq \Gnt{block}{\scriptsize (\ref{block})}  \glq\Gt{\}}\grq \\

Ein \emph{Block} besteht aus Typdefinitionen, Variablen- und Konstantendeklarationen sowie keinem bis mehreren Kommandos.
Der Gültigkeitsbereich dieser Deklarationen endet mit dem Ende des Blockes. Kommandos werden zur Laufzeit in Reihenfolge ihres
Auftretens ausgeführt.

Ein \kw{return}-Kommando führt zum sofortigen Verlassen der umschließenden Funktion.
Bei Funktionen mit einem anderen Rückgabetyp als \kw{void} muss ein \emph{Ausdruck} mit angegeben
werden, wobei der Rückgabewert der Funktion auf den Wert des Ausdrucks gesetzt wird.

\subsection{Ausdrücke}\label{Ausdruecke}
\Gnt{ausdruck}:\label{ausdruck}\\
\hspace*{1cm}\Gnt{asdr\_zuweisung}{\scriptsize (\ref{asdr_zuweisung})} \\

Der Ausdruck höchster Präzedenz ist eine Zuweisung, auf welche einfach in der Regel für
allgemeine Ausdrücke verwiesen wird.


\subsubsection{Zuweisung}\label{Zuweisung}
\Gnt{asdr\_zuweisung}:\label{asdr_zuweisung}\\
\hspace*{1cm}\Gnt{asdr\_ternaer}{\scriptsize (\ref{asdr_ternaer})} \\
\hspace*{1cm}\Gnt{asdr\_ternaer}{\scriptsize (\ref{asdr_ternaer})} \glq\Gt{=}\grq  \Gnt{ausdruck}{\scriptsize (\ref{ausdruck})} \\

Ein Zuweisungsausdruck \Gnt{asdr\_zuweisung}{\scriptsize (\ref{asdr_zuweisung})} weist dem Ausdruck der linken Seite den Wert des Ausdrucks
auf der rechten Seite zu. Sofern eine Zuweisung tatsächlich gegeben ist muss
der Ausdruck auf der linken Seite eine Variable oder ein Arrayelement
bzw. von diesen ein Swizzle-Attribut (siehe \ref{Vektorattribute}) bezeichnen.

Der Typ des Ausdrucks muss zuweisungskompatibel zum bezeichneten Wert
sein.

Ein Zuweisungsausdruck selbst hat den Wert des zugewiesenen bezeichneten Wertes nach der Zuweisung.

\subsubsection{Basisausdruck}\label{Basisausdruck}
\Gnt{asdr\_basis}:\label{asdr_basis}\\
\hspace*{1cm}\glq\Gt{(}\grq  \Gnt{ausdruck}{\scriptsize (\ref{ausdruck})}  \glq\Gt{)}\grq \\
\hspace*{1cm}\glq\Gt{(}\grq  \Gnt{ausdruck}{\scriptsize (\ref{ausdruck})}  \glq\Gt{)}\grq \Gnt{attribut\_liste\_oder\_array\_element}{\scriptsize (\ref{attribut_liste_oder_array_element})} \\
\hspace*{1cm}\Gnt{BEZEICHNER}{\scriptsize (\ref{BEZEICHNER})} \\
\hspace*{1cm}\Gnt{BEZEICHNER}{\scriptsize (\ref{BEZEICHNER})} \Gnt{attribut\_liste\_oder\_array\_element}{\scriptsize (\ref{attribut_liste_oder_array_element})} \\
\hspace*{1cm}\Gnt{asdr\_konst\_bool}{\scriptsize (\ref{asdr_konst_bool})} \\
\hspace*{1cm}\Gnt{asdr\_konst\_bool}{\scriptsize (\ref{asdr_konst_bool})} \Gnt{attribut\_liste\_oder\_array\_element}{\scriptsize (\ref{attribut_liste_oder_array_element})} \\
\hspace*{1cm}\Gnt{funktion\_aufruf}{\scriptsize (\ref{funktion_aufruf})} \\
\hspace*{1cm}\Gnt{funktion\_aufruf}{\scriptsize (\ref{funktion_aufruf})} \Gnt{attribut\_liste\_oder\_array\_element}{\scriptsize (\ref{attribut_liste_oder_array_element})} \\
\hspace*{1cm}\Gnt{NUMERIC}{\scriptsize (\ref{NUMERIC})} \\
\hspace*{1cm}\Gnt{NUMERIC}{\scriptsize (\ref{NUMERIC})} \Gnt{attribut\_liste\_oder\_array\_element}{\scriptsize (\ref{attribut_liste_oder_array_element})} \\
\Gnt{attribut\_liste\_oder\_array\_element}:\label{attribut_liste_oder_array_element}\\
\hspace*{1cm}\glq\Gt{.}\grq  \Gnt{attribut}{\scriptsize (\ref{attribut})} \\
\hspace*{1cm}\glq\Gt{.}\grq  \Gnt{attribut}{\scriptsize (\ref{attribut})} \Gnt{attribut\_liste\_oder\_array\_element}{\scriptsize (\ref{attribut_liste_oder_array_element})} \\
\hspace*{1cm}\glq\Gt{[}\grq  \Gnt{ausdruck}{\scriptsize (\ref{ausdruck})}  \glq\Gt{]}\grq \\
\hspace*{1cm}\glq\Gt{[}\grq  \Gnt{ausdruck}{\scriptsize (\ref{ausdruck})}  \glq\Gt{]}\grq \Gnt{attribut\_liste\_oder\_array\_element}{\scriptsize (\ref{attribut_liste_oder_array_element})} \\

\emph{Basisausdrücke} sind die Ausdrücke niedrigster Präzedenz. Dies sind neben Bezeichnern
für Variablen, Konstanten, und Funktionsaufrufe auch Funktionsaufrufe und geschachtelte Ausdrücke.
Weiterhin können Zugriffe auf Attribute (siehe \Gnt{attribut}{\scriptsize (\ref{attribut})}) bzw. Arrayelemente spezifiziert werden.


\paragraph{Zuweisungskompatibilität}\label{Zuweisungskompatibilitaet}

Ein Ausdruck ist \emph{zuweisungskompatibel} zu einem bezeichneten Wert wenn
Wert und Ausdruck den selben Typ besitzen oder eine implizite Typumwandlung
möglich ist.

Ein Ausdruck ist \emph{verlustfrei zuweisungskompatibel} zu einem bezeichneten Wert
wenn Wert und Ausdruck den selben Typ besitzen oder eine implizite Typumwandlung
ohne Präzisionsverlust möglich ist.

Siehe auch \ref{Typumwandlung}.


\subsubsection{Binäre Ausdrücke}\label{Binaere Ausdruecke}

Ein \emph{binärer Ausdruck} wendet eine arithmetische, bitweise, logische oder vergleichende Verknüpfung
(bestimmt durch den Operator) auf die Werte zweier Ausdrücke an.

Werden zwei Vektoren verknüpft entspricht dies einer Verknüpfung der individuellen Komponenten an der gleichen
Stelle der Vektoren (\attr{x} mit \attr{x}, \attr{y} mit \attr{y}, $\dots$). Zwei verknüpfte Vektoren müssen
die gleiche Komponentenanzahl besitzen.

Bei der Auswertung wird zunächst der linke, dann der rechte Operand ausgewertet, und an\-schließend die Verknüpfung angewendet.

\paragraph{Arithmetische Operatoren}\label{Arithmetische Operatoren}
~\\
\Gnt{asdr\_mult}:\label{asdr_mult}\\
\hspace*{1cm}\Gnt{asdr\_unaer}{\scriptsize (\ref{asdr_unaer})} \\
\hspace*{1cm}\Gnt{asdr\_unaer}{\scriptsize (\ref{asdr_unaer})} \glq\Gt{*}\grq  \Gnt{asdr\_mult}{\scriptsize (\ref{asdr_mult})} \\
\hspace*{1cm}\Gnt{asdr\_unaer}{\scriptsize (\ref{asdr_unaer})} \glq\Gt{/}\grq  \Gnt{asdr\_mult}{\scriptsize (\ref{asdr_mult})} \\
\hspace*{1cm}\Gnt{asdr\_unaer}{\scriptsize (\ref{asdr_unaer})} \glq\Gt{\%}\grq  \Gnt{asdr\_mult}{\scriptsize (\ref{asdr_mult})} \\
\Gnt{asdr\_add}:\label{asdr_add}\\
\hspace*{1cm}\Gnt{asdr\_mult}{\scriptsize (\ref{asdr_mult})} \\
\hspace*{1cm}\Gnt{asdr\_mult}{\scriptsize (\ref{asdr_mult})} \glq\Gt{+}\grq  \Gnt{asdr\_add}{\scriptsize (\ref{asdr_add})} \\
\hspace*{1cm}\Gnt{asdr\_mult}{\scriptsize (\ref{asdr_mult})} \glq\Gt{-}\grq  \Gnt{asdr\_add}{\scriptsize (\ref{asdr_add})} \\

\op{\plus} addiert rechten und linken Operanden.

\op{-} subtrahiert den rechten von dem linken Operanden.

\op{*} multipliziert rechten und linken Operanden.

\op{/} dividiert den linken durch den rechten Operanden.

\op{\%} ist der Divisionsrest der Division des linken Operators durch den rechten Operator.
% TODO Mod bei floats?

Beide Operanden müssen von einem Integer- oder Fließkommatyp (\Gnt{typ\_num}{\scriptsize (\ref{typ_num})}) oder einem
zu einem dazu zuweisungskompatiblen Typ sein.

Sind beide Operanden vom Typ \kw{unsigned int}, so ist der Typ des ausgewerteten Ausdrucks \kw{unsigned int}.

Ist ein Operand vom Typ \kw{int} und ein Operand vom Typ \kw{int}, \kw{unsigned int} oder
verlustfrei zuweisungskompatibel zu \kw{int}, so ist der Typ des ausgewerteten Ausdrucks \kw{int}.

Ist ein Operand von einem Fließkommatyp und ein Operand von einem Integer- oder Fließkommatyp
oder zuweisungskompatibel zu dem Fließkommatyp des anderen Operanden,
so ist der Typ des ausgewerteten Ausdrucks von einem Fließkommatyp.
%Die Präzision entspricht der höheren Präzision der beiden Operanden.


\subsubsection{Unäre Ausdrücke}\label{Unaere Ausdruecke}
\Gnt{asdr\_unaer}:\label{asdr_unaer}\\
\hspace*{1cm}\glq\Gt{$\sim$}\grq  \Gnt{asdr\_basis}{\scriptsize (\ref{asdr_basis})} \\
\hspace*{1cm}\glq\Gt{-}\grq  \Gnt{asdr\_basis}{\scriptsize (\ref{asdr_basis})} \\
\hspace*{1cm}\glq\Gt{!}\grq  \Gnt{asdr\_basis}{\scriptsize (\ref{asdr_basis})} \\
\hspace*{1cm}\Gnt{asdr\_basis}{\scriptsize (\ref{asdr_basis})} \\

Ein \emph{unärer Ausdruck} transformiert den Wert des Operanden.

Ein unärer Operator, auf einen Vektor angewendet, entspricht der Anwendung auf die individuellen Komponenten.

\op{-} negiert den Operanden. Der Operand muss von einem Integertyp oder Fließkommatyp sein.
Ist der Operand von einem Integertyp, so ist der Typ des unären Ausdrucks \kw{int}.
Ist der Operand von einem Fließkommatyp, so ist der Typ des unären Ausdrucks vom selben Typ.

\op{$\sim$} bitweise invertiert den Operanden. Der Operand muss von einem Integertyp sein. Der Typ des unären Ausdrucks ist vom selben Typ.

\op{!} logisch invertiert den Operanden. Der Operand muss vom Typ \kw{bool} sein. Der Typ des unären Ausdrucks ist vom Typ \kw{bool}.

\subsubsection{Ternärer Ausdruck}\label{Ternaerer Ausdruck}
\Gnt{asdr\_ternaer}:\label{asdr_ternaer}\\
\hspace*{1cm}\Gnt{asdr\_logisch\_oder}{\scriptsize (\ref{asdr_logisch_oder})} \\
\hspace*{1cm}\Gnt{asdr\_logisch\_oder}{\scriptsize (\ref{asdr_logisch_oder})} \glq\Gt{?}\grq  \Gnt{ausdruck}{\scriptsize (\ref{ausdruck})}  \glq\Gt{:}\grq  \Gnt{ausdruck}{\scriptsize (\ref{ausdruck})} \\

Die \emph{Bedingung} ist ein boole'scher Ausdruck. Ergibt sich dieser Ausdruck zu \kw{true},
so wird der \emph{Wahr-Ausdruck} (dem \kw{?} folgend) ausgewertet, und der Wert des ternären
Ausdrucks ergibt sich zu dem Wert des \emph{Wahr-Ausdrucks}.
Ergibt sich die \emph{Bedingung} zu \kw{false}, so wird der \emph{Falsch-Ausdruck} (dem \kw{:} folgend)
ausgewertet, und der Wert des ternären
Ausdrucks ergibt sich zu dem Wert des \emph{Falsch-Ausdrucks}.

\emph{Wahr-} und \emph{Falsch-Ausdruck} müssen vom gleichen Typ sein.


\subsubsection{Vergleichsoperatoren}\label{Vergleichsoperatoren}
\Gnt{asdr\_vergleich\_gleich}:\label{asdr_vergleich_gleich}\\
\hspace*{1cm}\Gnt{asdr\_vergleich\_ungleich}{\scriptsize (\ref{asdr_vergleich_ungleich})} \\
\hspace*{1cm}\Gnt{asdr\_vergleich\_ungleich}{\scriptsize (\ref{asdr_vergleich_ungleich})} \glq\Gt{!=}\grq  \Gnt{asdr\_vergleich\_gleich}{\scriptsize (\ref{asdr_vergleich_gleich})} \\
\hspace*{1cm}\Gnt{asdr\_vergleich\_ungleich}{\scriptsize (\ref{asdr_vergleich_ungleich})} \glq\Gt{==}\grq  \Gnt{asdr\_vergleich\_gleich}{\scriptsize (\ref{asdr_vergleich_gleich})} \\
\Gnt{asdr\_vergleich\_ungleich}:\label{asdr_vergleich_ungleich}\\
\hspace*{1cm}\Gnt{asdr\_add}{\scriptsize (\ref{asdr_add})} \\
\hspace*{1cm}\Gnt{asdr\_add}{\scriptsize (\ref{asdr_add})} \glq\Gt{>}\grq  \Gnt{asdr\_vergleich\_ungleich}{\scriptsize (\ref{asdr_vergleich_ungleich})} \\
\hspace*{1cm}\Gnt{asdr\_add}{\scriptsize (\ref{asdr_add})} \glq\Gt{>=}\grq  \Gnt{asdr\_vergleich\_ungleich}{\scriptsize (\ref{asdr_vergleich_ungleich})} \\
\hspace*{1cm}\Gnt{asdr\_add}{\scriptsize (\ref{asdr_add})} \glq\Gt{<}\grq  \Gnt{asdr\_vergleich\_ungleich}{\scriptsize (\ref{asdr_vergleich_ungleich})} \\
\hspace*{1cm}\Gnt{asdr\_add}{\scriptsize (\ref{asdr_add})} \glq\Gt{<=}\grq  \Gnt{asdr\_vergleich\_ungleich}{\scriptsize (\ref{asdr_vergleich_ungleich})} \\

\op{>} gibt an ob der linke Operand größer als der rechte Operand ist.

\op{>=} gibt an ob der linke Operand größer oder gleich dem der rechte Operand ist.

\op{<} gibt an ob der linke Operand kleiner als der rechte Operand ist.

\op{<=} gibt an ob der linke Operand kleiner oder gleich dem der rechte Operand ist.

\op{==} gibt an ob der linke Operand gleich dem der rechte Operand ist.

\op{!=} gibt an ob der linke Operand nicht gleich dem der rechte Operand ist.

Beide Operanden müssen von einem Integer- oder Fließkommatyp (\Gnt{typ\_num}{\scriptsize (\ref{typ_num})}) oder einem
dazu zuweisungskompatiblen Typ sein.

Sind beide Operanden vom Typ \kw{unsigned int} oder \kw{int}, so findet ein Vergleich von Ganzzahlwerten statt.

Ist ein Operand vom Typ \kw{int} und ein Operand vom Typ \kw{int}, \kw{unsigned int} oder
verlustfrei zuweisungskompatibel zu \kw{int} oder \kw{unsigned int}, so findet ein Vergleich von Ganzzahlwerten statt.

Ist ein Operand von einem Fließkommatyp und ein Operand von einem Integer- oder Fließkommatyp
oder zuweisungskompatibel zu dem Fließkommatyp des anderen Operanden,
so findet ein Vergleich von Fließkommawerten statt.
%Die Präzision entspricht der höheren Präzision der beiden Operanden.


\subsubsection{Logische Operatoren}\label{Logische Operatoren}
\Gnt{asdr\_logisch\_oder}:\label{asdr_logisch_oder}\\
\hspace*{1cm}\Gnt{asdr\_logisch\_und}{\scriptsize (\ref{asdr_logisch_und})} \\
\hspace*{1cm}\Gnt{asdr\_logisch\_und}{\scriptsize (\ref{asdr_logisch_und})} \glq\Gt{||}\grq  \Gnt{asdr\_logisch\_oder}{\scriptsize (\ref{asdr_logisch_oder})} \\
\Gnt{asdr\_logisch\_und}:\label{asdr_logisch_und}\\
\hspace*{1cm}\Gnt{asdr\_vergleich\_gleich}{\scriptsize (\ref{asdr_vergleich_gleich})} \\
\hspace*{1cm}\Gnt{asdr\_vergleich\_gleich}{\scriptsize (\ref{asdr_vergleich_gleich})} \glq\Gt{\&\&}\grq  \Gnt{asdr\_logisch\_und}{\scriptsize (\ref{asdr_logisch_und})} \\

%  "!" - unäres NOT

Beide Operanden müssen vom Typ \kw{bool} sein.

\op{\&\&} logisch UND-verknüpft die Operanden.

\op{||} logisch ODER-verknüpft die Operanden.


\subsubsection{Boole'sche Werte}\label{Boole'sche Werte}
\Gnt{asdr\_konst\_bool}:\label{asdr_konst_bool}\\
\hspace*{1cm}\glq\Gt{true}\grq \\
\hspace*{1cm}\glq\Gt{false}\grq \\

\kw{true} (wahr) und \kw{false} (unwahr) sind Werte vom Typ \kw{bool} (siehe \Gnt{typ\_bool}{\scriptsize (\ref{typ_bool})}).


\subsubsection{Attribute}\label{Attribute}
\Gnt{attribut}:\label{attribut}\\
\hspace*{1cm}\Gnt{BEZEICHNER}{\scriptsize (\ref{BEZEICHNER})} \\

% TODO 'Attr-Vektor', das Swizzle, Komp. ist
% TODO kein Bzch-Attribut sondern Attribute ausschreiben

Einige Variablen (und Konstanten) besitzen über den Wert hinaus weitere Eigenschaften, die
über Attribute abgefragt werden können.

\emph{Attributbezeichner} sind dabei keine reservierten
Schlüsselwörter, d.h. sie können auch als Variablenbezeichner, Funktionsbezeichner etc.
verwendet werden.

Vektorattribute werden in \ref{Vektorattribute} beschrieben.\\
Matrixattribute werden in \ref{Matrixattribute} beschrieben.\\
Arrayattribute werden in \ref{Arrayattribute} beschrieben.\\


\subsection{Typen}\label{Typen}
\Gnt{typ\_basis}:\label{typ_basis}\\
\hspace*{1cm}\Gnt{typ\_num}{\scriptsize (\ref{typ_num})} \\
\hspace*{1cm}\Gnt{typ\_bool}{\scriptsize (\ref{typ_bool})} \\
\hspace*{1cm}\Gnt{typ\_vektor}{\scriptsize (\ref{typ_vektor})} \\
\hspace*{1cm}\Gnt{typ\_matrix}{\scriptsize (\ref{typ_matrix})} \\
\hspace*{1cm}\Gnt{typ\_sampler}{\scriptsize (\ref{typ_sampler})} \\
\hspace*{1cm}\Gnt{BEZEICHNER}{\scriptsize (\ref{BEZEICHNER})} \\
\Gnt{typ}:\label{typ}\\
\hspace*{1cm}\Gnt{typ\_basis}{\scriptsize (\ref{typ_basis})} \\
\hspace*{1cm}\Gnt{typ\_basis}{\scriptsize (\ref{typ_basis})} \Gnt{typ\_array}{\scriptsize (\ref{typ_array})} \\
\subsubsection{Boole'scher Typ}\label{Boole'scher Typ}
\Gnt{typ\_bool}:\label{typ_bool}\\
\hspace*{1cm}\glq\Gt{bool}\grq \\

Der Typ \kw{bool} spezifiziert den boole'schen Typ mit den möglichen Werten
\kw{true} (wahr) oder \kw{false} (unwahr).


\subsubsection{Numerische Typen}\label{Numerische Typen}
\Gnt{typ\_num}:\label{typ_num}\\
\hspace*{1cm} \glq\Gt{int}\grq \\
\hspace*{1cm}\glq\Gt{unsigned}\grq  \glq\Gt{int}\grq \\
\hspace*{1cm}\glq\Gt{float}\grq \\

Der Typ \kw{int} spezifiziert vorzeichenbehaftete Ganzzahlwerte.

Der Typ \kw{unsigned int} spezifiziert vorzeichenlose Ganzzahlwerte.

Genauigkeit und Wertebereich sind jeweils implementierungsabhängig.

Der Typ \kw{float} spezifiziert Fließkommazahlen.

%Optional kann durch \Gnt{Typ-Float-Präz}{\scriptsize (\ref{Typ-Float-Präz})} eine gewünschte Präzision \Gnt{Bzch-Präzision}{\scriptsize (\ref{Bzch-Präzision})} angegeben werden.
%Dies kann beeinflussen, in welchem Format und/oder mit welcher Genauigkeit ein
%Wert gespeichert wird.
%\Gnt{Bzch-Präzision}{\scriptsize (\ref{Bzch-Präzision})} ist ein Bezeichner. Vordefinierte Präzisionen sind
%\kw{half}, \kw{single} und \kw{double}. Dies sollte einem Datenformat
%nach IEEE ??? in halber, einfacher und doppelter Genauigkeit entsprechen.

%Ist keine Präzision angegeben soll \kw{single} angenommen werden.

%Unbekannte Präzisionen sind keine Syntaxfehler; Programme mir unbekannten Präzisionen sind gültig.

%Präzision ist Teil des Typs.

%Präzision ist nicht verbindlich. Genauigkeit und Wertebereich sind jeweils implementierungsabhängig.


\subsubsection{Vektortypen}\label{Vektortypen}
\Gnt{typ\_vektor}:\label{typ_vektor}\\
\hspace*{1cm}\Gnt{typ\_vektor\_int}{\scriptsize (\ref{typ_vektor_int})} \\
\hspace*{1cm}\Gnt{typ\_vektor\_float}{\scriptsize (\ref{typ_vektor_float})} \\
\hspace*{1cm}\Gnt{typ\_vektor\_bool}{\scriptsize (\ref{typ_vektor_bool})} \\
\Gnt{typ\_vektor\_int}:\label{typ_vektor_int}\\
\hspace*{1cm} \glq\Gt{int1}\grq \\
\hspace*{1cm} \glq\Gt{int2}\grq \\
\hspace*{1cm} \glq\Gt{int3}\grq \\
\hspace*{1cm} \glq\Gt{int4}\grq \\
\hspace*{1cm}\glq\Gt{unsigned}\grq  \glq\Gt{int1}\grq \\
\hspace*{1cm}\glq\Gt{unsigned}\grq  \glq\Gt{int2}\grq \\
\hspace*{1cm}\glq\Gt{unsigned}\grq  \glq\Gt{int3}\grq \\
\hspace*{1cm}\glq\Gt{unsigned}\grq  \glq\Gt{int4}\grq \\
\Gnt{typ\_vektor\_float}:\label{typ_vektor_float}\\
\hspace*{1cm}\glq\Gt{float1}\grq \\
\hspace*{1cm}\glq\Gt{float2}\grq \\
\hspace*{1cm}\glq\Gt{float3}\grq \\
\hspace*{1cm}\glq\Gt{float4}\grq \\
\Gnt{typ\_vektor\_bool}:\label{typ_vektor_bool}\\
\hspace*{1cm}\glq\Gt{bool1}\grq \\
\hspace*{1cm}\glq\Gt{bool2}\grq \\
\hspace*{1cm}\glq\Gt{bool3}\grq \\
\hspace*{1cm}\glq\Gt{bool4}\grq \\

% "normalized"?


Der \emph{Basistyp} eines Vektors ist ein Integer-, Boole'scher oder Fließkommatyp.
Die Zahl $N$ nach dem Basistyp ist vom Typ
Ganzzahl, kann Werte zwischen 1 und 4 inklusive annehmen, und gibt die Komponentenanzahl an.
Ein Vektortyp spezifiziert ein $N$-Tupel von Werten
des Basistyps.


% Besserer Platz?

\paragraph{Vektorattribute}\label{Vektorattribute}
Einzelne Komponenten können entweder mit Array-Syntax oder den Attributen \attr{x},
\attr{y}, \attr{z}, \attr{w}, \attr{r}, \attr{g}, \attr{b}, \attr{a} angesprochen
werden. \attr{x}/\attr{r} spezifiert die erste Komponente,
\attr{y}/\attr{g} die zweite Komponente, \attr{z}/\attr{b} die dritte Komponente,
\attr{w}/\attr{a} die vierte Komponente. Ansprechen einer nicht existierenden Komponente
führt zu einem Fehler.

Weitere Attribute sind die "`Swizzle"'-Attribute (\Gnt{SWIZZLE\_RGBA}{\scriptsize (\ref{SWIZZLE_RGBA})}, \Gnt{SWIZZLE\_XYZW}{\scriptsize (\ref{SWIZZLE_XYZW})}).
Das Ergebnis eines Swizzles ist vom Typ eines Vektors des Basistyps, mit der Anzahl der Komponenten
entsprechend der Anzahl der Komponenten im "`Swizzle"'. Die erste Komponente des Wertes entspricht
der Komponente des Ursprungsvektors, die an erster Stelle des Swizzles identifiziert wird, usw.

\paragraph{Vektorkonstruktoren}\label{Vektorkonstruktoren}

Vektorkonstruktoren sind besondere, vorgegebene Funktionen, die gleiche Bezeichner wie Vektortypen haben
(siehe auch \Gnt{funktion\_aufruf}{\scriptsize (\ref{funktion_aufruf})}).

Wird ein Vektor aus einem \emph{Basisausdruck} konstruiert, so haben alle Komponenten diesen Wert.
Der \emph{Basisausdruck} muss als Typ einen Basistyp besitzen.

Wird ein Vektor aus mehreren Basis- oder Vektorausdrücken oder einem Vektorausdruck konstruiert,
so wird der ersten Komponente der Wert des ersten Basisausdrucks bzw. der ersten Vektorausdruckskomponente zugewiesen,
der zweiten Komponente der Wert des nächsten Basisausdrucks bzw. der nächsten Vektorausdruckskomponente usw.
Allen Komponenten muss ein Wert zugewiesen werden. Die Anzahl aller Basisausdrücke und Komponenten der Vektorausdrücke
zusammen muss exakt der Komponentenanzahl entsprechen.
Alle verwendeten Vektortypen müssen vom gleichen Basistyp wie der zu konstruierende Vektor sein.


\subsubsection{Matrixtypen}\label{Matrixtypen}
\Gnt{typ\_matrix}:\label{typ_matrix}\\
\hspace*{1cm}\Gnt{typ\_matrix\_float}{\scriptsize (\ref{typ_matrix_float})} \\
\hspace*{1cm}\Gnt{typ\_matrix\_int}{\scriptsize (\ref{typ_matrix_int})} \\
\hspace*{1cm}\Gnt{typ\_matrix\_bool}{\scriptsize (\ref{typ_matrix_bool})} \\
\Gnt{typ\_matrix\_float}:\label{typ_matrix_float}\\
\hspace*{1cm}\glq\Gt{float1x1}\grq \\
\hspace*{1cm}\glq\Gt{float2x1}\grq \\
\hspace*{1cm}\glq\Gt{float3x1}\grq \\
\hspace*{1cm}\glq\Gt{float4x1}\grq \\
\hspace*{1cm}\glq\Gt{float1x2}\grq \\
\hspace*{1cm}\glq\Gt{float2x2}\grq \\
\hspace*{1cm}\glq\Gt{float3x2}\grq \\
\hspace*{1cm}\glq\Gt{float4x2}\grq \\
\hspace*{1cm}\glq\Gt{float1x3}\grq \\
\hspace*{1cm}\glq\Gt{float2x3}\grq \\
\hspace*{1cm}\glq\Gt{float3x3}\grq \\
\hspace*{1cm}\glq\Gt{float4x3}\grq \\
\hspace*{1cm}\glq\Gt{float1x4}\grq \\
\hspace*{1cm}\glq\Gt{float2x4}\grq \\
\hspace*{1cm}\glq\Gt{float3x4}\grq \\
\hspace*{1cm}\glq\Gt{float4x4}\grq \\
\Gnt{typ\_matrix\_int}:\label{typ_matrix_int}\\
\hspace*{1cm} \glq\Gt{int1x1}\grq \\
\hspace*{1cm} \glq\Gt{int2x1}\grq \\
\hspace*{1cm} \glq\Gt{int3x1}\grq \\
\hspace*{1cm} \glq\Gt{int4x1}\grq \\
\hspace*{1cm} \glq\Gt{int1x2}\grq \\
\hspace*{1cm} \glq\Gt{int2x2}\grq \\
\hspace*{1cm} \glq\Gt{int3x2}\grq \\
\hspace*{1cm} \glq\Gt{int4x2}\grq \\
\hspace*{1cm} \glq\Gt{int1x3}\grq \\
\hspace*{1cm} \glq\Gt{int2x3}\grq \\
\hspace*{1cm} \glq\Gt{int3x3}\grq \\
\hspace*{1cm} \glq\Gt{int4x3}\grq \\
\hspace*{1cm} \glq\Gt{int1x4}\grq \\
\hspace*{1cm} \glq\Gt{int2x4}\grq \\
\hspace*{1cm} \glq\Gt{int3x4}\grq \\
\hspace*{1cm} \glq\Gt{int4x4}\grq \\
\hspace*{1cm}\glq\Gt{unsigned}\grq  \glq\Gt{int1x1}\grq \\
\hspace*{1cm}\glq\Gt{unsigned}\grq  \glq\Gt{int2x1}\grq \\
\hspace*{1cm}\glq\Gt{unsigned}\grq  \glq\Gt{int3x1}\grq \\
\hspace*{1cm}\glq\Gt{unsigned}\grq  \glq\Gt{int4x1}\grq \\
\hspace*{1cm}\glq\Gt{unsigned}\grq  \glq\Gt{int1x2}\grq \\
\hspace*{1cm}\glq\Gt{unsigned}\grq  \glq\Gt{int2x2}\grq \\
\hspace*{1cm}\glq\Gt{unsigned}\grq  \glq\Gt{int3x2}\grq \\
\hspace*{1cm}\glq\Gt{unsigned}\grq  \glq\Gt{int4x2}\grq \\
\hspace*{1cm}\glq\Gt{unsigned}\grq  \glq\Gt{int1x3}\grq \\
\hspace*{1cm}\glq\Gt{unsigned}\grq  \glq\Gt{int2x3}\grq \\
\hspace*{1cm}\glq\Gt{unsigned}\grq  \glq\Gt{int3x3}\grq \\
\hspace*{1cm}\glq\Gt{unsigned}\grq  \glq\Gt{int4x3}\grq \\
\hspace*{1cm}\glq\Gt{unsigned}\grq  \glq\Gt{int1x4}\grq \\
\hspace*{1cm}\glq\Gt{unsigned}\grq  \glq\Gt{int2x4}\grq \\
\hspace*{1cm}\glq\Gt{unsigned}\grq  \glq\Gt{int3x4}\grq \\
\hspace*{1cm}\glq\Gt{unsigned}\grq  \glq\Gt{int4x4}\grq \\
\Gnt{typ\_matrix\_bool}:\label{typ_matrix_bool}\\
\hspace*{1cm}\glq\Gt{bool1x1}\grq \\
\hspace*{1cm}\glq\Gt{bool2x1}\grq \\
\hspace*{1cm}\glq\Gt{bool3x1}\grq \\
\hspace*{1cm}\glq\Gt{bool4x1}\grq \\
\hspace*{1cm}\glq\Gt{bool1x2}\grq \\
\hspace*{1cm}\glq\Gt{bool2x2}\grq \\
\hspace*{1cm}\glq\Gt{bool3x2}\grq \\
\hspace*{1cm}\glq\Gt{bool4x2}\grq \\
\hspace*{1cm}\glq\Gt{bool1x3}\grq \\
\hspace*{1cm}\glq\Gt{bool2x3}\grq \\
\hspace*{1cm}\glq\Gt{bool3x3}\grq \\
\hspace*{1cm}\glq\Gt{bool4x3}\grq \\
\hspace*{1cm}\glq\Gt{bool1x4}\grq \\
\hspace*{1cm}\glq\Gt{bool2x4}\grq \\
\hspace*{1cm}\glq\Gt{bool3x4}\grq \\
\hspace*{1cm}\glq\Gt{bool4x4}\grq \\

Der \emph{Basistyp} einer Matrix ist ein Integer-, Boole'scher oder Fließkommatyp.
Der Basistyp wird von einer Zeichenkette der Form $N$\kw{x}$M$ gefolgt. $N$ und
$M$ sind vom Typ Ganzzahl, können jeweils Werte zwischen 1 und 4 inklusive annehmen,
und geben die Spalten- und Zeilenzahl
der Matrix an. Ein Matrixtyp spezifiziert eine  $N \times M$-Matrix von Werten des Basistyps.

\paragraph{Matrixattribute}\label{Matrixattribute}

Einzelne Zeilen können über das Attribute \attr{row} angesprochen werden. Dieses
verhält sich wie ein Array mit $M$ Elementen des Vektortyps \emph{Basistyp}$N$.

Einzelne Spalten können über das Attribut \attr{col} angesprochen werden. Dieses
verhält sich wie ein Array mit $N$ Elementen des \emph{Basistyp}$M$.

Das Attribut \attr{transposed} ist eine Matrix vom Typ \emph{Basistyp}$M$x$N$
und hat als Wert die transponierte Ursprungsmatrix.

Matrizen mit gleichem $N$ und $M$ besitzen das Attribut \attr{inverted}.
Es ist eine Matrix vom Typ \emph{Basistyp}$N$x$M$ und hat als Wert die invertierte Ursprungsmatrix.
Das Ergebnis ist undefiniert, wenn die Ursprungsmatrix nicht invertierbar ist.

\paragraph{Matrixkonstruktoren}\label{Matrixkonstruktoren}

Matrixkonstruktoren sind besondere, vorgegebene Funktionen die gleiche Bezeichner wie Matrixtypen haben
(siehe auch \Gnt{funktion\_aufruf}{\scriptsize (\ref{funktion_aufruf})}).

%Wird eine Matrix aus einem Matrix-Ausdruck konstruiert so haben alle Felder der Matrix die
%Werte der entsprechenden Quellmatrix. Die Dimensionen der Quellmatrix muss denen von
%\Gnt{Typ-Matrix}{\scriptsize (\ref{Typ-Matrix})} entsprechen.
%\Gnt{Asdr-Basis}{\scriptsize (\ref{Asdr-Basis})} muss als Typ einen Matrix-Typ besitzen.

Eine Matrix wird aus einem oder mehreren Vektorausdrücken konstruiert. Den Zeilen werden
jeweils die Werte der Vektorausdrücke zugewiesen. Die Komponentenanzahl der Vektoren muss dabei
$N$ sein. Die Anzahl der gegebenen Vektoren muss $M$ sein.

%Wird eine Matrix aus einem oder mehreren Basisausdrücken konstruiert so werden den Feldern
%nacheinander jeweils die Werte dieser Ausdrücke zugewiesen. Den Feldern der ersten Zeile werden
%die ersten \Gnt{N}{\scriptsize (\ref{N})} Werte zugewiesen, den Feldern der zweiten Zeile die nächsten \Gnt{N}{\scriptsize (\ref{N})} Werte
%usw. Die Anzahl der gegebenen Werten muss \Gnt{N}{\scriptsize (\ref{N})}$\cdot$\Gnt{M}{\scriptsize (\ref{M})} sein.

\subsubsection{Samplertypen}\label{Samplertypen}
\Gnt{typ\_sampler}:\label{typ_sampler}\\
\hspace*{1cm}\glq\Gt{sampler1D}\grq \\
\hspace*{1cm}\glq\Gt{sampler2D}\grq \\
\hspace*{1cm}\glq\Gt{sampler3D}\grq \\
\hspace*{1cm}\glq\Gt{samplerCUBE}\grq \\

Samplertypen repräsentieren Textureinheiten der Hardware. Texturen verschiedener
Dimensionalität müssen in Shadingprogrammen verschiedenartig angesprochen werden;
dies bedingt die mehrfachen Samplertypen, die jeweils einen speziellen Texturtyp
reflektieren.

Texturen werden über spezielle vordefinierte Funktionen ausgelesen (siehe~\ref{Texturfunktionen}).


\subsubsection{Arraytypen}\label{Arraytypen}
\Gnt{typ\_array}:\label{typ_array}\\
\hspace*{1cm}\glq\Gt{[}\grq  \glq\Gt{]}\grq \\
\hspace*{1cm}\glq\Gt{[}\grq  \glq\Gt{]}\grq \Gnt{typ\_array}{\scriptsize (\ref{typ_array})} \\

Der Typ der Elemente im Arraytyp wird durch den \emph{Basistyp} spezifiziert (siehe auch \Gnt{typ}{\scriptsize (\ref{typ})}).
Die Anzahl der Elemente in individuellen Variablen oder Konstanten eines Arraytyps
ergibt sich aus der Anzahl der Element des zugewiesenen Array-Wertes.

Elemente einer Array-Variable oder -Konstante können mit \ident{Bezeichner[}$\mathit{Index}$\ident{]} angesprochen
werden. Das erste Element wird mit dem Wert $0$ für $\mathit{Index}$ angesprochen.
Liegt $\mathit{Index}$ ausserhalb des gültigen Bereiches ($0\dots{}N-1$) so ist der ausgelesene Wert
undefiniert.


\paragraph{Arrayattribute}\label{Arrayattribute}

Arrays besitzen ein Attribut \attr{length} vom Typ \kw{unsigned int} dessen Wert
die Anzahl der Elemente im Array ist.


\paragraph{Arraykonstruktoren}\label{Arraykonstruktoren}

%Ktor-Array:
%  Bzch-Typ."[]" "(" [Ausdruck { "," Ausdruck } "," ] ")"

Arraykonstruktoren sind besondere, vorgegebene Funktionen die gleiche Bezeichner wie Arraytypen haben
(siehe auch \Gnt{funktion\_aufruf}{\scriptsize (\ref{funktion_aufruf})}).

Ein Array wird aus keinem, einem oder mehreren Werten des Elementtyps konstruiert.
Die Anzahl der Elemente entspricht der Anzahl der dem Konstruktor übergebenen Ausdrücke.

\subsubsection{Typdefinitionen}\label{Typdefinitionen}
\Gnt{typ\_definition}:\label{typ_definition}\\
\hspace*{1cm}\glq\Gt{typedef}\grq  \Gnt{typ}{\scriptsize (\ref{typ})}  \Gnt{BEZEICHNER}{\scriptsize (\ref{BEZEICHNER})} \\

\Gnt{typ\_definition}{\scriptsize (\ref{typ_definition})} deklariert einen neuen Typ mit dem angegeben Bezeichner der ein Alias für den durch \glq\Gnt{typ}\grq bezeichneten Typen ist.

Die Sichtbarkeit des Typbezeichners beginnt hinter der \Gnt{typ\_definition}{\scriptsize (\ref{typ_definition})}
und reicht bis zum Ende des umgebenden Gültigkeitsbereiches.

Der Bezeichner darf keinen anderen Bezeichner des umgebenden Gültigkeitsbereiches überdecken.


\subsubsection{Typumwandlung}\label{Typumwandlung}

\paragraph{{Implizite Typumwandlungen}}\label{__Implizite Typumwandlungen__}
Ausdrücke eines numerischen Typs können in einen anderen numerischen Typ umgewandelt werden.
Wird ein Ausdruck einen numerischen Typs einer Variable oder einem formalen Funktionsparameter
eines anderen numerischen Typs zugewiesen, findet eine \emph{implizite} Typumwandlung statt
(Zuweisungskompatibilität).

Ist \emph{verlustfreie} Zuweisungskompatibilität verlangt, wird eine implizite Typumwandlung nur
vorgenommen, wenn kein Präzisionsverlust auftritt (nach nachfolgenden Definitionen). Tritt ein
Präzisionsverlust auf, ist das Programm ungültig.

\kw{int} und \kw{unsigned int} werden als immer untereinander ohne Präzisionsverlust zuweisbar
angenommen.

Eine Zuweisung von \kw{int} oder \kw{unsigned int} an \kw{float} wird immer als ohne Präzisionsverlust zuweisbar
angenommen.

Eine Zuweisung von \kw{float} zu \kw{int} oder \kw{unsigned int} wird immer mit Präzisionsverlust zuweisbar angenommen.

\paragraph{{Explizite Typumwandlungen}}\label{__Explizite Typumwandlungen__}
Ausdrücke können explizit in einen anderen Typ umgewandelt werden (syntaktisch ein Funktionsaufruf mit
Typbezeichner als Funktionsbezeichner). Umwandlungen mit Präzisionsverlust sind dabei erlaubt.
Ein Ausdruck einer expliziten Typumwandlung hat den Typ in den explizit umgewandelt wurde.


\subsection{Funktionen}\label{Funktionen}
\Gnt{dekl\_funktion}:\label{dekl_funktion}\\
\hspace*{1cm}\Gnt{typ\_funktion}{\scriptsize (\ref{typ_funktion})}  \Gnt{BEZEICHNER}{\scriptsize (\ref{BEZEICHNER})}  \glq\Gt{(}\grq  \glq\Gt{)}\grq  \glq\Gt{\{}\grq  \Gnt{block}{\scriptsize (\ref{block})}  \glq\Gt{\}}\grq \\
\hspace*{1cm}\Gnt{typ\_funktion}{\scriptsize (\ref{typ_funktion})}  \Gnt{BEZEICHNER}{\scriptsize (\ref{BEZEICHNER})}  \glq\Gt{(}\grq \Gnt{funktion\_param\_formal}{\scriptsize (\ref{funktion_param_formal})}  \glq\Gt{)}\grq  \glq\Gt{\{}\grq  \Gnt{block}{\scriptsize (\ref{block})}  \glq\Gt{\}}\grq \\
\Gnt{typ\_funktion}:\label{typ_funktion}\\
\hspace*{1cm}\Gnt{typ}{\scriptsize (\ref{typ})} \\
\hspace*{1cm}\glq\Gt{void}\grq \\
\Gnt{funktion\_param\_formal}:\label{funktion_param_formal}\\
\hspace*{1cm} \Gnt{typ}{\scriptsize (\ref{typ})}  \Gnt{BEZEICHNER}{\scriptsize (\ref{BEZEICHNER})} \\
\hspace*{1cm} \Gnt{typ}{\scriptsize (\ref{typ})}  \Gnt{BEZEICHNER}{\scriptsize (\ref{BEZEICHNER})} \glq\Gt{,}\grq  \Gnt{funktion\_param\_formal}{\scriptsize (\ref{funktion_param_formal})} \\
\hspace*{1cm} \Gnt{typ}{\scriptsize (\ref{typ})}  \Gnt{BEZEICHNER}{\scriptsize (\ref{BEZEICHNER})} \glq\Gt{=}\grq  \Gnt{ausdruck}{\scriptsize (\ref{ausdruck})} \\
\hspace*{1cm} \Gnt{typ}{\scriptsize (\ref{typ})}  \Gnt{BEZEICHNER}{\scriptsize (\ref{BEZEICHNER})} \glq\Gt{=}\grq  \Gnt{ausdruck}{\scriptsize (\ref{ausdruck})} \glq\Gt{,}\grq  \Gnt{funktion\_param\_formal}{\scriptsize (\ref{funktion_param_formal})} \\
\hspace*{1cm}\glq\Gt{out}\grq  \Gnt{typ}{\scriptsize (\ref{typ})}  \Gnt{BEZEICHNER}{\scriptsize (\ref{BEZEICHNER})} \\
\hspace*{1cm}\glq\Gt{out}\grq  \Gnt{typ}{\scriptsize (\ref{typ})}  \Gnt{BEZEICHNER}{\scriptsize (\ref{BEZEICHNER})} \glq\Gt{,}\grq  \Gnt{funktion\_param\_formal}{\scriptsize (\ref{funktion_param_formal})} \\
\hspace*{1cm}\glq\Gt{out}\grq  \Gnt{typ}{\scriptsize (\ref{typ})}  \Gnt{BEZEICHNER}{\scriptsize (\ref{BEZEICHNER})} \glq\Gt{=}\grq  \Gnt{ausdruck}{\scriptsize (\ref{ausdruck})} \\
\hspace*{1cm}\glq\Gt{out}\grq  \Gnt{typ}{\scriptsize (\ref{typ})}  \Gnt{BEZEICHNER}{\scriptsize (\ref{BEZEICHNER})} \glq\Gt{=}\grq  \Gnt{ausdruck}{\scriptsize (\ref{ausdruck})} \glq\Gt{,}\grq  \Gnt{funktion\_param\_formal}{\scriptsize (\ref{funktion_param_formal})} \\
\hspace*{1cm}\glq\Gt{in}\grq  \Gnt{typ}{\scriptsize (\ref{typ})}  \Gnt{BEZEICHNER}{\scriptsize (\ref{BEZEICHNER})} \\
\hspace*{1cm}\glq\Gt{in}\grq  \Gnt{typ}{\scriptsize (\ref{typ})}  \Gnt{BEZEICHNER}{\scriptsize (\ref{BEZEICHNER})} \glq\Gt{,}\grq  \Gnt{funktion\_param\_formal}{\scriptsize (\ref{funktion_param_formal})} \\
\hspace*{1cm}\glq\Gt{in}\grq  \Gnt{typ}{\scriptsize (\ref{typ})}  \Gnt{BEZEICHNER}{\scriptsize (\ref{BEZEICHNER})} \glq\Gt{=}\grq  \Gnt{ausdruck}{\scriptsize (\ref{ausdruck})} \\
\hspace*{1cm}\glq\Gt{in}\grq  \Gnt{typ}{\scriptsize (\ref{typ})}  \Gnt{BEZEICHNER}{\scriptsize (\ref{BEZEICHNER})} \glq\Gt{=}\grq  \Gnt{ausdruck}{\scriptsize (\ref{ausdruck})} \glq\Gt{,}\grq  \Gnt{funktion\_param\_formal}{\scriptsize (\ref{funktion_param_formal})} \\
\hspace*{1cm}\glq\Gt{in}\grq \glq\Gt{out}\grq  \Gnt{typ}{\scriptsize (\ref{typ})}  \Gnt{BEZEICHNER}{\scriptsize (\ref{BEZEICHNER})} \\
\hspace*{1cm}\glq\Gt{in}\grq \glq\Gt{out}\grq  \Gnt{typ}{\scriptsize (\ref{typ})}  \Gnt{BEZEICHNER}{\scriptsize (\ref{BEZEICHNER})} \glq\Gt{,}\grq  \Gnt{funktion\_param\_formal}{\scriptsize (\ref{funktion_param_formal})} \\
\hspace*{1cm}\glq\Gt{in}\grq \glq\Gt{out}\grq  \Gnt{typ}{\scriptsize (\ref{typ})}  \Gnt{BEZEICHNER}{\scriptsize (\ref{BEZEICHNER})} \glq\Gt{=}\grq  \Gnt{ausdruck}{\scriptsize (\ref{ausdruck})} \\
\hspace*{1cm}\glq\Gt{in}\grq \glq\Gt{out}\grq  \Gnt{typ}{\scriptsize (\ref{typ})}  \Gnt{BEZEICHNER}{\scriptsize (\ref{BEZEICHNER})} \glq\Gt{=}\grq  \Gnt{ausdruck}{\scriptsize (\ref{ausdruck})} \glq\Gt{,}\grq  \Gnt{funktion\_param\_formal}{\scriptsize (\ref{funktion_param_formal})} \\
\Gnt{funktion\_aufruf}:\label{funktion_aufruf}\\
\hspace*{1cm}\Gnt{typ}{\scriptsize (\ref{typ})}  \glq\Gt{(}\grq  \glq\Gt{)}\grq \\
\hspace*{1cm}\Gnt{typ}{\scriptsize (\ref{typ})}  \glq\Gt{(}\grq \Gnt{funktion\_param\_tatsaechl}{\scriptsize (\ref{funktion_param_tatsaechl})}  \glq\Gt{)}\grq \\
\Gnt{funktion\_param\_tatsaechl}:\label{funktion_param_tatsaechl}\\
\hspace*{1cm}\Gnt{ausdruck}{\scriptsize (\ref{ausdruck})} \\
\hspace*{1cm}\Gnt{ausdruck}{\scriptsize (\ref{ausdruck})} \glq\Gt{,}\grq  \Gnt{funktion\_param\_tatsaechl}{\scriptsize (\ref{funktion_param_tatsaechl})} \\

% TODO: Default-Parameter

\Gnt{dekl\_funktion}{\scriptsize (\ref{dekl_funktion})} deklariert eine Funktion mit dem gegebenen Bezeichner.

Die Sichtbarkeit des Funktionsbezeichners beginnt hinter dem \kw{)} von \Gnt{dekl\_funktion}{\scriptsize (\ref{dekl_funktion})}
und reicht bis zum Ende des Programms.

Der Bezeichner darf keinen anderen Bezeichner des globalen Gültigkeitsbereiches überdecken.
Ausnahme: Mehrere Funktionen können den gleichen Bezeicher besitzen; es handelt sich hierbei
um \emph{überladene} Funktionen (siehe~\ref{Ueberladene Funktionen}).

\Gnt{typ\_funktion}{\scriptsize (\ref{typ_funktion})} spezifiziert den Typ des Rückgabewertes der Funktion. Der spezielle Typ \kw{void}
gibt an, dass die Funktion keinen Wert zurückgibt.
%\kw{void} ist weder von, noch nach, irgendeinen anderen Typ umwandelbar.
Weder ist der Typ \kw{void} in irgendeinen anderen Typ umwandelbar, noch ist eine Umwandlung
von irgendeinem Typen nach \kw{void} möglich.

\Gnt{funktion\_param\_formal}{\scriptsize (\ref{funktion_param_formal})} beschreibt die Parameter, welche die Funktion annimmt.
Parameter können Eingabe- und/oder Ausgabeparameter sein: auf \emph{Eingabeparameter} kann
nur lesend zugegriffen werden, ein Schreibzugriff ist nicht erlaubt.
\emph{Ausgabeparameter} können beschrieben werden und jede Änderung wirkt sich auch sofort auf
den zugeordneten tatsächlichen Parameter aus. Ausgabeparameter können auch gelesen werden,
allerdings ist ihr Wert vor dem ersten Beschreiben undefiniert.
\emph{Ein- und Ausgabeparameter} können jederzeit gelesen werden, jeder Schreibzugriff wirkt sich
aber auch hier auf den zugeordneten tatsächlichen Parameter aus.

Funktionen werden mit \Gnt{funktion\_aufruf}{\scriptsize (\ref{funktion_aufruf})} aufgerufen. Zunächst werden die
gegebenen Werteparameter(\Gnt{funktion\_param\_tatsaechl}{\scriptsize (\ref{funktion_param_tatsaechl})}) den Funktionsparametern der Funktion über ihre Stelle zugeordnet.
Basierend auf der Anzahl der Parameter und den Typen der Parameterausdrücke wird eine Überladung der Funktion ausgewählt.
Den Funktionsparametern werden die entsprechenden Werte der Parameterausdrücke zugewiesen.
Die Parameterausdrücke werden von links nach rechts ausgewertet.
Die Programmausführung springt dann zu dem ersten
Kommando des Funktionsblockes. Nach Beendigung von dessen Ausführung wird dem Wert des Aufrufausdrucks der
Rückgabewert der Funktion zugewiesen. Die Programmausführung fährt mit dem dem ursprünglichen Funktionsaufruf
unmittelbar folgendem Kommando fort.

Als Bezeichner für die aufzurufende Funktion können nicht nur Bezeichner benutzerdefinierter und
vordefinierter Funktionen sondern auch \emph{Typbezeichner} dienen.
Im Fall von Vektor-, Matrix- oder Arraytypen sind dies die speziellen Konstruktoren
(siehe \ref{Vektorkonstruktoren}, \ref{Matrixkonstruktoren} und \ref{Arraykonstruktoren}).
Im Fall von boole'schen oder numerischen Typen wird eine explizite Typumwandlung (siehe
\ref{Typumwandlung}) veranlasst.


\subsubsection{Überladene Funktionen}\label{Ueberladene Funktionen}

Es können mehrere Funktionen mit identischem Bezeichner, aber unterschiedlicher Signatur,
deklariert werden. Die Signatur einer Funktion wird aus Position und Typ jedes Parameters
bestimmt. Werden zwei Funktionen mit identischem Bezeichner und Signatur deklariert ist
das Programm ungültig.

Die aufzurufende Überladung einer Funktion wird über die Signatur bestimmt. Existiert
eine Überladung, deren Signatur genau mit den übergebenen tatsächlichen Parametern
übereinstimmt wird diese ausgewählt. Andernfalls werden aus den Überladungen mit
gleicher Parameteranzahl, aber unterschiedlichen Typen "`Kandidaten"' für den
Aufruf ausgewählt. Eine Funktion ist "`Kandidat"', wenn an jeder Stelle der Signatur
der Typ identisch mit dem übergebenen Typ oder in diesen verlustfrei umwandelbar ist.
Gibt es genau einen Kandidaten wird dieser ausgewählt und die Programmausführung
mit diesem fortgesetzt. Gibt es keinen oder mehrere Kandidaten ist das Programm
ungültig.


\subsection{Konstanten und Variablen}\label{Konstanten und Variablen}
\subsubsection{Variablendeklarationen}\label{Variablendeklarationen}
\Gnt{dekl\_var}:\label{dekl_var}\\
\hspace*{1cm}\Gnt{typ}{\scriptsize (\ref{typ})}  \Gnt{dekl\_var\_bzch\_init\_liste}{\scriptsize (\ref{dekl_var_bzch_init_liste})} \\
\Gnt{dekl\_var\_bzch\_init\_liste}:\label{dekl_var_bzch_init_liste}\\
\hspace*{1cm}\Gnt{dekl\_var\_bzch\_init}{\scriptsize (\ref{dekl_var_bzch_init})} \\
\hspace*{1cm}\Gnt{dekl\_var\_bzch\_init}{\scriptsize (\ref{dekl_var_bzch_init})} \glq\Gt{,}\grq  \Gnt{dekl\_var\_bzch\_init\_liste}{\scriptsize (\ref{dekl_var_bzch_init_liste})} \\
\Gnt{dekl\_var\_bzch\_init}:\label{dekl_var_bzch_init}\\
\hspace*{1cm}\Gnt{BEZEICHNER}{\scriptsize (\ref{BEZEICHNER})} \\
\hspace*{1cm}\Gnt{BEZEICHNER}{\scriptsize (\ref{BEZEICHNER})} \glq\Gt{=}\grq  \Gnt{ausdruck}{\scriptsize (\ref{ausdruck})} \\

\Gnt{dekl\_var}{\scriptsize (\ref{dekl_var})} deklariert eine oder mehrere neue Variablen des gegebenen Typs und den gegebenen Bezeichnern.

Die Sichtbarkeit der Bezeichner beginnt nach der jeweiligen Deklaration
und reicht bis zum Ende des umgebenden Blockes.

Der Bezeichner darf keinen anderen Bezeichner des umgebenden Gültigkeitsbereiches überdecken.

Der Wert von einer deklarierten Variable ist anfänglich undefiniert.

Wird ein initialer Ausdruck angegeben, so entspricht dies semantisch einer Zuweisungsoperation
hinter \Gnt{dekl\_var\_bzch\_init}{\scriptsize (\ref{dekl_var_bzch_init})}, aber vor der Deklaration der nächsten Variable.
Der Typ des initialen Ausdrucks muss zuweisungskompatibel zum Typ der Variable sein.

\subsubsection{Konstantendeklarationen}\label{Konstantendeklarationen}
\Gnt{dekl\_konst}:\label{dekl_konst}\\
\hspace*{1cm}\glq\Gt{const}\grq  \Gnt{typ}{\scriptsize (\ref{typ})}  \Gnt{dekl\_konst\_bzch\_init\_liste}{\scriptsize (\ref{dekl_konst_bzch_init_liste})} \\
\Gnt{dekl\_konst\_bzch\_init\_liste}:\label{dekl_konst_bzch_init_liste}\\
\hspace*{1cm}\Gnt{dekl\_konst\_bzch\_init}{\scriptsize (\ref{dekl_konst_bzch_init})} \\
\hspace*{1cm}\Gnt{dekl\_konst\_bzch\_init}{\scriptsize (\ref{dekl_konst_bzch_init})} \glq\Gt{,}\grq  \Gnt{dekl\_konst\_bzch\_init\_liste}{\scriptsize (\ref{dekl_konst_bzch_init_liste})} \\
\Gnt{dekl\_konst\_bzch\_init}:\label{dekl_konst_bzch_init}\\
\hspace*{1cm}\Gnt{BEZEICHNER}{\scriptsize (\ref{BEZEICHNER})}  \glq\Gt{=}\grq  \Gnt{ausdruck}{\scriptsize (\ref{ausdruck})} \\

\Gnt{dekl\_konst}{\scriptsize (\ref{dekl_konst})} deklariert eine oder mehrere neue Konstanten des gegebenen Typs und den gegebenen Bezeichnern.

Die Sichtbarkeit der Bezeichner beginnt nach der jeweiligen Deklaration
und reicht bis zum Ende des umgebenden Blockes.

Der Bezeichner darf keinen anderen Bezeichner des umgebenden Gültigkeitsbereiches überdecken.

Die Konstante wird mit dem Wert des initialen Ausdrucks initialisiert.
Der Typ des initialen Ausdrucks muss zuweisungskompatibel zum Typ der Konstante sein.

\subsection{Ablaufsteuerung}\label{Ablaufsteuerung}
\subsubsection{Verzweigungen}\label{Verzweigungen}
\Gnt{verzweigung}:\label{verzweigung}\\
\hspace*{1cm}\glq\Gt{if}\grq  \glq\Gt{(}\grq  \Gnt{ausdruck}{\scriptsize (\ref{ausdruck})}  \glq\Gt{)}\grq  \glq\Gt{\{}\grq  \Gnt{block}{\scriptsize (\ref{block})}  \glq\Gt{\}}\grq \\
\hspace*{1cm}\glq\Gt{if}\grq  \glq\Gt{(}\grq  \Gnt{ausdruck}{\scriptsize (\ref{ausdruck})}  \glq\Gt{)}\grq  \glq\Gt{\{}\grq  \Gnt{block}{\scriptsize (\ref{block})}  \glq\Gt{\}}\grq \glq\Gt{else}\grq  \glq\Gt{\{}\grq  \Gnt{block}{\scriptsize (\ref{block})}  \glq\Gt{\}}\grq \\

\glq\Gnt{ausdruck}\grq ist ein boole'scher Ausdruck, die \emph{Bedingung}.

Wenn die Bedingung zu \kw{true} ausgewertet wird, werden die Kommandos des ersten Blockes (\emph{Wahr-Block}) ausgeführt,
ansonsten die Kommandos des Blockes im \kw{else}-Zweig (\emph{Falsch-Block}).

Der \kw{else}-Zweig ist optional.

Die Berechnungsfrequenz der Verzweigung ist die kleinste gemeinsame Frequenz der Blöcke sowie der Bedingung.

\subsubsection{\kw{for}-Schleifen}\label{$_backslash$kw__for__-Schleifen}
\Gnt{schleife\_for}:\label{schleife_for}\\
\hspace*{1cm}\glq\Gt{for}\grq  \glq\Gt{(}\grq  \glq\Gt{;}\grq  \glq\Gt{;}\grq  \glq\Gt{)}\grq  \glq\Gt{\{}\grq  \Gnt{block}{\scriptsize (\ref{block})}  \glq\Gt{\}}\grq \\
\hspace*{1cm}\glq\Gt{for}\grq  \glq\Gt{(}\grq  \glq\Gt{;}\grq  \glq\Gt{;}\grq \Gnt{ausdruck}{\scriptsize (\ref{ausdruck})}  \glq\Gt{)}\grq  \glq\Gt{\{}\grq  \Gnt{block}{\scriptsize (\ref{block})}  \glq\Gt{\}}\grq \\
\hspace*{1cm}\glq\Gt{for}\grq  \glq\Gt{(}\grq  \glq\Gt{;}\grq \Gnt{ausdruck}{\scriptsize (\ref{ausdruck})}  \glq\Gt{;}\grq  \glq\Gt{)}\grq  \glq\Gt{\{}\grq  \Gnt{block}{\scriptsize (\ref{block})}  \glq\Gt{\}}\grq \\
\hspace*{1cm}\glq\Gt{for}\grq  \glq\Gt{(}\grq  \glq\Gt{;}\grq \Gnt{ausdruck}{\scriptsize (\ref{ausdruck})}  \glq\Gt{;}\grq \Gnt{ausdruck}{\scriptsize (\ref{ausdruck})}  \glq\Gt{)}\grq  \glq\Gt{\{}\grq  \Gnt{block}{\scriptsize (\ref{block})}  \glq\Gt{\}}\grq \\
\hspace*{1cm}\glq\Gt{for}\grq  \glq\Gt{(}\grq \Gnt{ausdruck}{\scriptsize (\ref{ausdruck})}  \glq\Gt{;}\grq  \glq\Gt{;}\grq  \glq\Gt{)}\grq  \glq\Gt{\{}\grq  \Gnt{block}{\scriptsize (\ref{block})}  \glq\Gt{\}}\grq \\
\hspace*{1cm}\glq\Gt{for}\grq  \glq\Gt{(}\grq \Gnt{ausdruck}{\scriptsize (\ref{ausdruck})}  \glq\Gt{;}\grq  \glq\Gt{;}\grq \Gnt{ausdruck}{\scriptsize (\ref{ausdruck})}  \glq\Gt{)}\grq  \glq\Gt{\{}\grq  \Gnt{block}{\scriptsize (\ref{block})}  \glq\Gt{\}}\grq \\
\hspace*{1cm}\glq\Gt{for}\grq  \glq\Gt{(}\grq \Gnt{ausdruck}{\scriptsize (\ref{ausdruck})}  \glq\Gt{;}\grq \Gnt{ausdruck}{\scriptsize (\ref{ausdruck})}  \glq\Gt{;}\grq  \glq\Gt{)}\grq  \glq\Gt{\{}\grq  \Gnt{block}{\scriptsize (\ref{block})}  \glq\Gt{\}}\grq \\
\hspace*{1cm}\glq\Gt{for}\grq  \glq\Gt{(}\grq \Gnt{ausdruck}{\scriptsize (\ref{ausdruck})}  \glq\Gt{;}\grq \Gnt{ausdruck}{\scriptsize (\ref{ausdruck})}  \glq\Gt{;}\grq \Gnt{ausdruck}{\scriptsize (\ref{ausdruck})}  \glq\Gt{)}\grq  \glq\Gt{\{}\grq  \Gnt{block}{\scriptsize (\ref{block})}  \glq\Gt{\}}\grq \\

Der erste Ausdruck ist die \emph{Initialisierung}. Der zweite Ausdruck ist die \emph{Schleifenbedingung}. Der dritte Ausdruck ist der \emph{Zählausdruck}.

Bei der Ausführung wird zunächst die \emph{Initialisierung} ausgeführt.

Die \emph{Schleifenbedingung} wird ausgewertet. Ist das Ergebnis \kw{true}, so werden die Kommandos des Schleifenblockes
und anschließend der \emph{Zählausdruck} ausgeführt.
Dieser Ablauf wird wiederholt bis eine Auswertung der \emph{Schleifenbedingung} das Ergebnis \kw{false} hat.

Die Berechnungsfrequenz der Schleife ist die kleinste gemeinsame Frequenz des Schleifenblockes sowie
der \emph{Initialisierung}, der \emph{Schleifenbedingung} und des \emph{Zählausdrucks}.

\subsubsection{\kw{while}-Schleifen}\label{$_backslash$kw__while__-Schleifen}
\Gnt{schleife\_while}:\label{schleife_while}\\
\hspace*{1cm}\glq\Gt{while}\grq  \glq\Gt{(}\grq  \Gnt{ausdruck}{\scriptsize (\ref{ausdruck})}  \glq\Gt{)}\grq  \glq\Gt{\{}\grq  \Gnt{block}{\scriptsize (\ref{block})}  \glq\Gt{\}}\grq \\

\glq\Gnt{ausdruck}\grq ist ein boole'scher Ausdruck, die \emph{Bedingung}.

Die Bedingung wird ausgewertet. Ist das Ergebnis \kw{true}, so werden die Kommandos des Schleifenblocks
ausgeführt. Dieser Ablauf wird wiederholt bis eine Auswertung der Bedingung das Ergebnis \kw{false} hat.

Die Berechnungsfrequenz der Schleife ist die kleinste gemeinsame Frequenz des Schleifenblockes sowie der Bedingung.

\subsection{Lexikalische Einheiten}\label{Lexikalische Einheiten}

\emph{Anmerkung:} Da diese Regeln die lexikalischen Einheiten selbst beschreiben gibt es
hier \emph{keine} weiteren Trennzeichen zwischen Terminalen!


\subsubsection{Bezeichner}\label{Bezeichner}
\Gnt{Number}:\label{Number}\\
\hspace*{1cm}Unicode-Ziffer\\
\Gnt{Letter}:\label{Letter}\\
\hspace*{1cm}Unicode-Buchstabe\\
\Gnt{Comp\_XYZW}:\label{Comp_XYZW}\\
\hspace*{1cm}\glq\Gt{x}\grq \\
\hspace*{1cm}\glq\Gt{y}\grq \\
\hspace*{1cm}\glq\Gt{z}\grq \\
\hspace*{1cm}\glq\Gt{w}\grq \\
\Gnt{Comp\_RGBA}:\label{Comp_RGBA}\\
\hspace*{1cm}\glq\Gt{r}\grq \\
\hspace*{1cm}\glq\Gt{g}\grq \\
\hspace*{1cm}\glq\Gt{b}\grq \\
\hspace*{1cm}\glq\Gt{a}\grq \\
\Gnt{SWIZZLE\_XYZW}:\label{SWIZZLE_XYZW}\\
\hspace*{1cm}\Gnt{Comp\_XYZW}{\scriptsize (\ref{Comp_XYZW})} \\
\hspace*{1cm}\Gnt{Comp\_XYZW}{\scriptsize (\ref{Comp_XYZW})} \Gnt{Comp\_XYZW}{\scriptsize (\ref{Comp_XYZW})} \\
\hspace*{1cm}\Gnt{Comp\_XYZW}{\scriptsize (\ref{Comp_XYZW})} \Gnt{Comp\_XYZW}{\scriptsize (\ref{Comp_XYZW})} \Gnt{Comp\_XYZW}{\scriptsize (\ref{Comp_XYZW})} \\
\hspace*{1cm}\Gnt{Comp\_XYZW}{\scriptsize (\ref{Comp_XYZW})} \Gnt{Comp\_XYZW}{\scriptsize (\ref{Comp_XYZW})} \Gnt{Comp\_XYZW}{\scriptsize (\ref{Comp_XYZW})} \Gnt{Comp\_XYZW}{\scriptsize (\ref{Comp_XYZW})} \\
\Gnt{SWIZZLE\_RGBA}:\label{SWIZZLE_RGBA}\\
\hspace*{1cm}\Gnt{Comp\_RGBA}{\scriptsize (\ref{Comp_RGBA})} \\
\hspace*{1cm}\Gnt{Comp\_RGBA}{\scriptsize (\ref{Comp_RGBA})} \Gnt{Comp\_RGBA}{\scriptsize (\ref{Comp_RGBA})} \\
\hspace*{1cm}\Gnt{Comp\_RGBA}{\scriptsize (\ref{Comp_RGBA})} \Gnt{Comp\_RGBA}{\scriptsize (\ref{Comp_RGBA})} \Gnt{Comp\_RGBA}{\scriptsize (\ref{Comp_RGBA})} \\
\hspace*{1cm}\Gnt{Comp\_RGBA}{\scriptsize (\ref{Comp_RGBA})} \Gnt{Comp\_RGBA}{\scriptsize (\ref{Comp_RGBA})} \Gnt{Comp\_RGBA}{\scriptsize (\ref{Comp_RGBA})} \Gnt{Comp\_RGBA}{\scriptsize (\ref{Comp_RGBA})} \\
\Gnt{BEZEICHNER}:\label{BEZEICHNER}\\
\hspace*{1cm}\Gnt{Letter}{\scriptsize (\ref{Letter})} \\
\hspace*{1cm}\Gnt{Letter}{\scriptsize (\ref{Letter})} \Gnt{LetterNumberStr}{\scriptsize (\ref{LetterNumberStr})} \\
\hspace*{1cm}\glq\Gt{\_}\grq \\
\hspace*{1cm}\glq\Gt{\_}\grq \Gnt{LetterNumberStr}{\scriptsize (\ref{LetterNumberStr})} \\
\Gnt{LetterNumberStr}:\label{LetterNumberStr}\\
\hspace*{1cm}\Gnt{Letter}{\scriptsize (\ref{Letter})} \\
\hspace*{1cm}\Gnt{Letter}{\scriptsize (\ref{Letter})} \Gnt{LetterNumberStr}{\scriptsize (\ref{LetterNumberStr})} \\
\hspace*{1cm}\Gnt{Number}{\scriptsize (\ref{Number})} \\
\hspace*{1cm}\Gnt{Number}{\scriptsize (\ref{Number})} \Gnt{LetterNumberStr}{\scriptsize (\ref{LetterNumberStr})} \\
\hspace*{1cm}\glq\Gt{\_}\grq \\
\hspace*{1cm}\glq\Gt{\_}\grq \Gnt{LetterNumberStr}{\scriptsize (\ref{LetterNumberStr})} \\
\Gnt{Whitespace}:\label{Whitespace}\\
\hspace*{1cm}Unicode-Separator\\
\Gnt{WS}:\label{WS}\\
\hspace*{1cm}\Gnt{Whitespace}{\scriptsize (\ref{Whitespace})} \\
\hspace*{1cm}\Gnt{Whitespace}{\scriptsize (\ref{Whitespace})} \Gnt{WS}{\scriptsize (\ref{WS})} \\

\emph{Bezeicher} ergeben sich aus ein oder mehreren Buchstaben, Ziffern und dem Zeichen \glq\kw{\_}\grq. Das erste Zeichen
darf keine Ziffer sein.

Der Unterteilung in Buchstaben und Ziffern liegen Unicode-Kategorien zugrunde (siehe~\cite{unicode}, Abschn. 4.5).
% TODO Ref: Unicode-Standard - 5.1.0 4.1

Zwei Bezeichner sind identisch wenn deren Zeichensequenzen \emph{kanonisch äquivalent} nach Unicode sind (siehe~\cite{unicode}, Kap. 3, Def. D70).
% Unicode 5.1.0 2.12

Lexikalische Einheiten können mit einem oder mehreren Zeichen der Unicode-Kategorie "`Separator"' getrennt werden.


\subsubsection{Numerische Literale}\label{Numerische Literale}
\Gnt{Numbers}:\label{Numbers}\\
\hspace*{1cm}\Gnt{Number}{\scriptsize (\ref{Number})} \\
\hspace*{1cm}\Gnt{Number}{\scriptsize (\ref{Number})} \Gnt{Numbers}{\scriptsize (\ref{Numbers})} \\
\Gnt{Numbers\_Sign}:\label{Numbers_Sign}\\
\hspace*{1cm} \Gnt{Numbers}{\scriptsize (\ref{Numbers})} \\
\hspace*{1cm}\glq\Gt{-}\grq  \Gnt{Numbers}{\scriptsize (\ref{Numbers})} \\
\Gnt{Numbers\_Frac}:\label{Numbers_Frac}\\
\hspace*{1cm}\glq\Gt{.}\grq  \Gnt{Numbers}{\scriptsize (\ref{Numbers})} \\
\Gnt{Num\_Float\_Exp}:\label{Num_Float_Exp}\\
\hspace*{1cm}\glq\Gt{e}\grq  \Gnt{Numbers\_Sign}{\scriptsize (\ref{Numbers_Sign})} \\
\hspace*{1cm}\glq\Gt{E}\grq  \Gnt{Numbers\_Sign}{\scriptsize (\ref{Numbers_Sign})} \\
\Gnt{Num\_Float}:\label{Num_Float}\\
\hspace*{1cm}\Gnt{Numbers}{\scriptsize (\ref{Numbers})} \\
\hspace*{1cm}\Gnt{Numbers}{\scriptsize (\ref{Numbers})} \Gnt{Num\_Float\_Exp}{\scriptsize (\ref{Num_Float_Exp})} \\
\hspace*{1cm}\Gnt{Numbers}{\scriptsize (\ref{Numbers})} \glq\Gt{.}\grq \\
\hspace*{1cm}\Gnt{Numbers}{\scriptsize (\ref{Numbers})} \glq\Gt{.}\grq \Gnt{Num\_Float\_Exp}{\scriptsize (\ref{Num_Float_Exp})} \\
\hspace*{1cm}\Gnt{Numbers}{\scriptsize (\ref{Numbers})} \Gnt{Numbers\_Frac}{\scriptsize (\ref{Numbers_Frac})} \\
\hspace*{1cm}\Gnt{Numbers}{\scriptsize (\ref{Numbers})} \Gnt{Numbers\_Frac}{\scriptsize (\ref{Numbers_Frac})} \Gnt{Num\_Float\_Exp}{\scriptsize (\ref{Num_Float_Exp})} \\
\hspace*{1cm}\Gnt{Numbers\_Frac}{\scriptsize (\ref{Numbers_Frac})} \\
\hspace*{1cm}\Gnt{Numbers\_Frac}{\scriptsize (\ref{Numbers_Frac})} \Gnt{Num\_Float\_Exp}{\scriptsize (\ref{Num_Float_Exp})} \\
\Gnt{Digit\_Hex}:\label{Digit_Hex}\\
\hspace*{1cm}\glq\kw{0}\grq..\glq\kw{9}\grq\\
\hspace*{1cm}\glq\kw{A}\grq..\glq\kw{F}\grq\\
\hspace*{1cm}\glq\kw{a}\grq..\glq\kw{f}\grq\\
\Gnt{Digits\_Hex}:\label{Digits_Hex}\\
\hspace*{1cm}\Gnt{Digit\_Hex}{\scriptsize (\ref{Digit_Hex})} \\
\hspace*{1cm}\Gnt{Digit\_Hex}{\scriptsize (\ref{Digit_Hex})} \Gnt{Digits\_Hex}{\scriptsize (\ref{Digits_Hex})} \\
\Gnt{Num\_Hex}:\label{Num_Hex}\\
\hspace*{1cm}\glq\Gt{0x}\grq  \Gnt{Digits\_Hex}{\scriptsize (\ref{Digits_Hex})} \\
\Gnt{NUMERIC}:\label{NUMERIC}\\
\hspace*{1cm}\Gnt{Num\_Float}{\scriptsize (\ref{Num_Float})} \\
\hspace*{1cm}\Gnt{Num\_Hex}{\scriptsize (\ref{Num_Hex})} \\

Ganzzahlen können in Dezimal- und Hexadezimalnotation angegeben werden.

Fließkommazahlen können als Dezimalbruch oder in Exponentialschreibweise
angegeben werden.


\subsubsection{Kommentare}\label{Kommentare}
\Gnt{COMMENT}:\label{COMMENT}\\
\hspace*{1cm}\glq\Gt{//}\grq  \glq\Gt{$\backslash$r}\grq \\
\hspace*{1cm}\glq\Gt{//}\grq  \glq\Gt{$\backslash$n}\grq \\
\Gnt{ML\_COMMENT}:\label{ML_COMMENT}\\
\hspace*{1cm}\glq\kwMlcStart\grq beliebige Zeichen \glq\kwMlcEnd\grq\\

\Gnt{COMMENT}{\scriptsize (\ref{COMMENT})} ist ein einzeiliger Kommentar. Alle Eingabezeichen bis zum nächsten Zeilenumbruch
werden ignoriert.

\Gnt{ML\_COMMENT}{\scriptsize (\ref{ML_COMMENT})} ist ein mehrzeiliger Kommentar. Alle Eingabezeichen, inklusive Zeilenumbrüchen,
zwischen Start- (\kwMlcStart) und nächster Endmarkierung (\kwMlcEnd) werden ignoriert.



\section{Standardumgebung}

\subsection{Vordefinierte Funktionen}

\subsubsection{Skalarprodukt}

\texttt{\emph{<VektorN>} dot (\emph{<VektorN>} a, \emph{<VektorN} b);}

\ident{dot} berechnet das Skalarprodukt zweier Vektoren. \ident{\emph{<VektorN>}}
muss dabei ein Vektor eines numerischen Typs sein (siehe~\ref{typ_vektor_int}).

\subsubsection{Vektorprodukt}
\texttt{float3 cross (float3 a, float3 b);}\\
\texttt{int3 cross (int3 a, int3 b);}\\
\texttt{unsigned int3 cross (unsigned int3 a, unsigned int3 b);}

\ident{cross} berechnet das Vektorprodukt zweier Dreikomponentenvektoren.

\subsubsection{Matrixmultiplikation}

\texttt{\emph{<MatrixLxN>} mul (\emph{<MatrixLxM>} a, \emph{<MatrixMxN>} b);}\\
\texttt{\emph{<VektorN>} mul (\emph{<VektorM>} a, \emph{<MatrixMxN>} b);}\\
\texttt{\emph{<VektorL>} mul (\emph{<MatrixLxM>} a, \emph{<VektorM>} b);}

\ident{mul} multipliziert zwei Matrizen miteinander. Der Basistyp aller
übergebenen Matrizen muss dabei gleich sein. Nur numerische Basistypen
sind erlaubt.

Als Sonderfall ist es möglich, einen Vektor an Stelle eines der Matrixparameter zu
übergeben. Das Ergebnis ist äquivalent zu einer Matrixmultiplikation bei der
aus dem Vektor eine einzeilige bzw. einspaltige Matrix erstellt wurde. Das
Ergebnis ist eine einzeilige bzw. einspaltige Matrix, diese eine Zeile bzw.
Spalte wird als Vektor zurückgegeben. Der Basistyp der übergebenen Matrix
und des Vektors muss dabei gleich sein. Nur numerische Basistypen sind erlaubt.

\subsubsection{Normalisierung}

\texttt{\emph{<VektorN>} normalize (\emph{<VektorN>} v);}

\ident{normalize} gibt den Einheitsvektor zurück, der in die gleiche Richtung wie
\ident{v} zeigt. Das Ergebnis ist undefiniert wenn \ident{v} die Länge $0$ hat.
\ident{\emph{<VektorN>}} muss dabei ein Vektor eines numerischen Typs sein.

\subsubsection{Euklidische Länge}

\texttt{float length (\emph{<VektorN>} v);}

\ident{length} gibt die euklidische Länge des Vektors \ident{v} zurück.
\ident{\emph{<VektorN>}} muss dabei ein Vektor eines numerischen Typs sein.

\subsubsection{Texturfunktionen}
\label{Texturfunktionen}

\texttt{float4 tex1D (sampler1D tex, \emph{<Vektor1>} coord);}\\
\texttt{float4 tex2D (sampler2D tex, \emph{<Vektor2>} coord);}\\
\texttt{float4 tex3D (sampler3D tex, \emph{<Vektor3>} coord);}\\
\texttt{float4 texCUBE (samplerCUBE tex, \emph{<Vektor3>} coord);}

Die \ident{tex}*-Funktionen veranlassen das Auslesen der \ident{tex} entsprechenden
Textureinheit mit den Koordinaten \ident{coord}.

\ident{coord} muss dabei ein Vektor eines numerischen Typs und der angegebenen
Komponentenzahl sein.

Texturen verschiedener
Dimensionalität müssen in Shadingprogrammen verschiedenartig angesprochen werden;
dies bedingt die verschiedenen Texturfunktionen, die jeweils einen speziellen Texturtyp
reflektieren.

\subsubsection{Minimum}

\texttt{\emph{<Basistyp>} min (\emph{<Basistyp>} a, \emph{<Basistyp>} b);}\\
\texttt{\emph{<VektorN>} min (\emph{<VektorN>} a, \emph{<VektorN>} b);}

\ident{min} gibt das Minimum zweier numerischer Werte bzw. das komponentenweise
Minimum zweier numerischer Vektoren zurück (analog der Funktion binärer Ausdrücke,
siehe~\ref{Binaere Ausdruecke}).

\subsubsection{Maximum}

\texttt{\emph{<Basistyp>} max (\emph{<Basistyp>} a, \emph{<Basistyp>} b);}\\
\texttt{\emph{<VektorN>} max (\emph{<VektorN>} a, \emph{<VektorN>} b);}

\ident{max} gibt das Maximum zweier numerischer Werte bzw. das komponentenweise
Maximum zweier numerischer Vektoren zurück (analog der Funktion binärer Ausdrücke,
siehe~\ref{Binaere Ausdruecke}).

\subsubsection{Potenz}

\texttt{float pow (float basis, float exp);}\\

\ident{pow} gibt den für $\mathit{basis}^\mathit{exp}$ berechneten Wert zurück.


\section{Ausblick}

\cleardoublepage
\appendix
\bibliography{thesis_de}
\addcontentsline{toc}{section}{Literatur}

\end{document}
